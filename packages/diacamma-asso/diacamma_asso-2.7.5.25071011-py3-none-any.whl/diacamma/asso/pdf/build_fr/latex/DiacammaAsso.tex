%% Generated by Sphinx.
\def\sphinxdocclass{report}
\documentclass[a4paper,10pt,oneside,french]{sphinxmanual}
\ifdefined\pdfpxdimen
   \let\sphinxpxdimen\pdfpxdimen\else\newdimen\sphinxpxdimen
\fi \sphinxpxdimen=.75bp\relax
\ifdefined\pdfimageresolution
    \pdfimageresolution= \numexpr \dimexpr1in\relax/\sphinxpxdimen\relax
\fi
%% let collapsible pdf bookmarks panel have high depth per default
\PassOptionsToPackage{bookmarksdepth=5}{hyperref}

\PassOptionsToPackage{booktabs}{sphinx}
\PassOptionsToPackage{colorrows}{sphinx}

\PassOptionsToPackage{warn}{textcomp}
\usepackage[utf8]{inputenc}
\ifdefined\DeclareUnicodeCharacter
% support both utf8 and utf8x syntaxes
  \ifdefined\DeclareUnicodeCharacterAsOptional
    \def\sphinxDUC#1{\DeclareUnicodeCharacter{"#1}}
  \else
    \let\sphinxDUC\DeclareUnicodeCharacter
  \fi
  \sphinxDUC{00A0}{\nobreakspace}
  \sphinxDUC{2500}{\sphinxunichar{2500}}
  \sphinxDUC{2502}{\sphinxunichar{2502}}
  \sphinxDUC{2514}{\sphinxunichar{2514}}
  \sphinxDUC{251C}{\sphinxunichar{251C}}
  \sphinxDUC{2572}{\textbackslash}
\fi
\usepackage{cmap}
\usepackage[T1]{fontenc}
\usepackage{amsmath,amssymb,amstext}
\usepackage{babel}



\usepackage{tgtermes}
\usepackage{tgheros}
\renewcommand{\ttdefault}{txtt}



\usepackage[Sonny]{fncychap}
\ChNameVar{\Large\normalfont\sffamily}
\ChTitleVar{\Large\normalfont\sffamily}
\usepackage{sphinx}

\fvset{fontsize=auto}
\usepackage{geometry}


% Include hyperref last.
\usepackage{hyperref}
% Fix anchor placement for figures with captions.
\usepackage{hypcap}% it must be loaded after hyperref.
% Set up styles of URL: it should be placed after hyperref.
\urlstyle{same}


\usepackage{sphinxmessages}
\setcounter{tocdepth}{3}
\setcounter{secnumdepth}{3}


\title{Diacamma Asso}
\date{juil. 21, 2025}
\release{2.7.5}
\author{sd-libre}
\newcommand{\sphinxlogo}{\sphinxincludegraphics{DiacammaAsso.jpg}\par}
\renewcommand{\releasename}{Version}
\makeindex
\begin{document}

\ifdefined\shorthandoff
  \ifnum\catcode`\=\string=\active\shorthandoff{=}\fi
  \ifnum\catcode`\"=\active\shorthandoff{"}\fi
\fi

\pagestyle{empty}
\sphinxmaketitle
\pagestyle{plain}
\sphinxtableofcontents
\pagestyle{normal}
\phantomsection\label{\detokenize{index::doc}}


\sphinxstepscope


\chapter{Diacamma Asso}
\label{\detokenize{asso/index:diacamma-asso}}\label{\detokenize{asso/index::doc}}
\sphinxAtStartPar
Présentation du logiciel Diacamma Asso.

\sphinxstepscope


\section{Présentation}
\label{\detokenize{asso/presentation:presentation}}\label{\detokenize{asso/presentation::doc}}

\subsection{Description}
\label{\detokenize{asso/presentation:description}}
\sphinxAtStartPar
\sphinxstyleemphasis{Diacamma Asso} est un logiciel de gestion spécialement conçu pour les associations sportives ou culturelles.
Avec \sphinxstyleemphasis{Diacamma Asso}, donnez à votre association le logiciel qu’elle mérite. Pas besoin d’être informaticien pour avoir les outils adaptés à votre cas.

\sphinxAtStartPar
L’application de base est entièrement gratuite et vous permet de gérer les accès à vos données et au carnet d’adresses de votre association.

\sphinxAtStartPar
Les différents modules disponibles vous permettront, par exemple, de :
\begin{itemize}
\item {} 
\sphinxAtStartPar
Gérer plus spécifiquement vos adhérents avec leurs licences sportives et le suivi de leur parcours.

\item {} 
\sphinxAtStartPar
Gérer vos documents de façon centralisée grâce à la gestion documentaire.

\item {} 
\sphinxAtStartPar
Gérer la comptabilité de votre association.

\item {} 
\sphinxAtStartPar
Gérer vos devis, factures et factures proformat pour les adhérents subventionnés par un CE, entre autre.

\item {} 
\sphinxAtStartPar
Gérer des stocks de produit, lié à votre facturation et proposer une centrale d’achat à vos adhérents.

\end{itemize}

\sphinxAtStartPar
Ce manuel vous aidera dans l’utilisation de ce logiciel.
Si malgré tout, vous ne trouvez pas la réponse à vos problèmes, visiter notre site \sphinxurl{https://www.diacamma.org} où vous trouverez des tutoriels et des astuces.


\subsection{Installation}
\label{\detokenize{asso/presentation:installation}}
\sphinxAtStartPar
Vous pouvez installer \sphinxstyleemphasis{Diacamma Asso} sur un ordinateur dédié à votre association que ce soit un Apple Macintosh (macOS 10.12 « Sierra » et +) ou bien un PC sous MS\sphinxhyphen{}Windows (10 et +) ou sous GNU Linux (Ubuntu 22.04 ou +).

\sphinxAtStartPar
\sphinxstyleemphasis{Diacamma Asso} est un logiciel client/serveur : vous pouvez l’installer sur un ordinateur centralisateur et accéder aux données depuis un autre PC connecté au premier, sans limite du nombre d’utilisateurs simultanés.
Si le PC contenant les données est connecté de manière permanente à internet, vous aurez accès à vos données depuis n’importe où dans le monde!

\sphinxAtStartPar
Cette organisation est particulièrement intéressante pour permettre à plusieurs cadres associatifs d’avoir accès à des données communes.

\sphinxAtStartPar
Quel responsable ne s’est pas arraché les cheveux suite à un échange de documents via une clef USB où certaines modifications importantes se perdent?

\sphinxAtStartPar
Pour plus d’information, visiter notre site \sphinxurl{https://www.diacamma.org}


\subsection{Aides et support}
\label{\detokenize{asso/presentation:aides-et-support}}
\sphinxAtStartPar
Sur le site officiel du logiciel, \sphinxurl{https://www.diacamma.org}, où vous trouverez des tutoriels et un forum pour échanger des astuces entre utilisateurs.

\sphinxAtStartPar
Si vous souhaitez un service et un support plus personnalisé, vous pouvez faire confiance à notre partenaire officiel SLETO.
Pour en savoir plus sur des solutions d’hébergement et de support, rendez vous sur \sphinxurl{https://www.sleto.net}

\sphinxstepscope


\section{Prise en main}
\label{\detokenize{asso/first_step:prise-en-main}}\label{\detokenize{asso/first_step::doc}}
\sphinxAtStartPar
Le logiciel \sphinxstyleemphasis{Diacamma Asso} comprend un grand nombre de paramétrages et peut paraître difficile à configurer aux besoins de votre structure.

\sphinxAtStartPar
Nous vous proposons cette explication pour vous aider à franchir cette première étape dans l’utilisation de cet outil.
Un assistant de configuration reprend ces différentes étapes pour vous aider également à configurer ce logiciel.

\sphinxAtStartPar
Suivez pas à pas les différentes phases de réglages. Dans chaque étape, nous ne détaillons pas les fonctionnalités.
Nous vous invitons à vous référer au reste du manuel utilisateur pour cela.

\sphinxAtStartPar
Il peut être intéressant de réaliser des sauvegardes au cours de cette procédure.
Cela vous permettra, si vous faite une erreur, de revenir à une étape précédente sans tout recommencer (installation comprise).


\subsection{Vérifiez la mise à jour de votre logiciel}
\label{\detokenize{asso/first_step:verifiez-la-mise-a-jour-de-votre-logiciel}}
\sphinxAtStartPar
Commencez par vérifiez que votre logiciel est à jour.
En effet, nous diffusons régulièrement des correctifs qui ne sont pas toujours inclus dans les installateurs.


\subsection{Présentation de votre structures}
\label{\detokenize{asso/first_step:presentation-de-votre-structures}}\begin{quote}

\sphinxAtStartPar
Menu \sphinxstyleemphasis{Général/Nos coordonnées}
\end{quote}

\sphinxAtStartPar
Dans cette écran, vous pouvez décrire les coordonnées de votre structure.
De nombreuses fonctionnalités utilisent ces informations en particulier pour les impressions (facture, comptabilité, listing d’adhérents…)


\subsection{Gestion des adhérents : la saison et les cotisations}
\label{\detokenize{asso/first_step:gestion-des-adherents-la-saison-et-les-cotisations}}\begin{quote}

\sphinxAtStartPar
Menu \sphinxstyleemphasis{Administration/Modules (conf.)/Les saisons et les cotisations}
\end{quote}

\sphinxAtStartPar
Vérifiez que vous êtes sur la bonne saison active. Vous pouvez également changer les périodes associées à cette saison et lui associer des documents demandés liés à toutes nouvelles cotisations.
En début de saison, vous pouvez modifier la liste des différentes cotisations que vous proposez à vos adhérents.
Si vous utilisez le module de facturation, l’outil vous permettra d’associer à une cotisation des articles et un montant qui sera utilisé automatiquement.


\subsection{Gestion des adhérents : activités et équipes}
\label{\detokenize{asso/first_step:gestion-des-adherents-activites-et-equipes}}\begin{quote}

\sphinxAtStartPar
Menu \sphinxstyleemphasis{Administration/Modules (conf.)/Catégories}
\end{quote}

\sphinxAtStartPar
Dans le cas où vous voulez gérer plusieurs activités, équipes ou trier vos adhérents par classe d’âges, veuillez configurer ces différentes listes.


\subsection{Réglage de votre facturier}
\label{\detokenize{asso/first_step:reglage-de-votre-facturier}}\begin{quote}

\sphinxAtStartPar
Menu \sphinxstyleemphasis{Administration/Modules (conf.)/Configuration du règlement}

\sphinxAtStartPar
Menu \sphinxstyleemphasis{Administration/Modules (conf.)/Configuration commerciale du facturier}

\sphinxAtStartPar
Menu \sphinxstyleemphasis{Administration/Modules (conf.)/Configuration financière du facturier}
\end{quote}

\sphinxAtStartPar
Vérifiez que les paramètres du facturier vous correspondent.
Vous pouvez aussi ici, configurer le RIB de votre compte bancaire principal ainsi qu’en ajouter d’autres comptes courants.
Ces derniers vous seront utiles pour la fonctionnalité « dépôt de chèques » du facturier.


\subsection{Création de votre premier exercice comptable}
\label{\detokenize{asso/first_step:creation-de-votre-premier-exercice-comptable}}\begin{quote}

\sphinxAtStartPar
Menu \sphinxstyleemphasis{Administration/Modules (conf.)/Configuration comptables}

\sphinxAtStartPar
Menu \sphinxstyleemphasis{Comptabilité/Gestion comptable/plan comptable}
\end{quote}

\sphinxAtStartPar
Ouvrer votre premier exercice comptable et rendez le actif.
Vous devrez aussi créer le plan comptable de cet exercice pour avoir une comptabilité pleinement opérationnelle.


\subsection{Ajout des adhérents de l’association}
\label{\detokenize{asso/first_step:ajout-des-adherents-de-l-association}}\begin{quote}

\sphinxAtStartPar
Menu \sphinxstyleemphasis{Association/Adhérents/Adhérents cotisants}

\sphinxAtStartPar
Menu \sphinxstyleemphasis{Administration/Modules (conf.)/Importation de contacts}
\end{quote}

\sphinxAtStartPar
Il vous faut maintenant ajouter les différents adhérents de votre association.
Deux solutions pour cela: un par un ou via une importation.
Pour les ajouter un par un, vous devez ajouter une fiche pour chacun de vos adhérents.
Si vous avez une liste d’adhérents à importer, extrayez\sphinxhyphen{}la au format CSV, ODS ou XLS, en respectant la structure de fichier expliquée dans le manuel.

\sphinxAtStartPar
N’oubliez pas de leur attribuer une cotisation, sinon ils n’apparaîtrons pas dans la liste des adhérents cotisants.
En cas d’erreur, vous pouvez facilement les retrouver à l’aide de l’outil de recherche.

\sphinxAtStartPar
Notez que pour chaque nouvelle cotisation saisie, une facture non\sphinxhyphen{}validée sera générée automatiquement.


\subsection{Mise à jour comptable}
\label{\detokenize{asso/first_step:mise-a-jour-comptable}}\begin{quote}

\sphinxAtStartPar
Menu \sphinxstyleemphasis{Comptabilité/Gestion comptable/écritures comptables}

\sphinxAtStartPar
Menu \sphinxstyleemphasis{Comptabilité/Gestion comptable/Modèles d’écriture}
\end{quote}

\sphinxAtStartPar
Si vous mettez en place \sphinxstyleemphasis{Diacamma Asso} au cours de votre exercice comptable, vous devrez également saisir les à\sphinxhyphen{}nouveaux et le report\sphinxhyphen{}à\sphinxhyphen{}nouveau de l’exercice comptable précédent ainsi que saisir les écritures du début d’année.
Attention: n’oubliez pas que l’ajout d’une cotisation d’adhérent génère une facture ainsi que des écritures comptables associées. Tenez\sphinxhyphen{}en compte dans la reprise de votre comptabilité.
Pour vous aidez dans la saisie de votre comptabilité, nous vous conseillons d’utiliser les modèles d’écritures. Enregistrez en tant que modèles les écritures récurrentes que vous avez au cours d’une année. Ainsi vous pouvez rapidement compléter votre comptabilité en quelques clics.


\subsection{Le courriel}
\label{\detokenize{asso/first_step:le-courriel}}\begin{quote}

\sphinxAtStartPar
Menu \sphinxstyleemphasis{Administration/Modules (conf.)/Paramètrages de courriel}
\end{quote}

\sphinxAtStartPar
Définissez vos réglages pour votre courriel.
Le serveur smpt permettra à \sphinxstyleemphasis{Diacamma Asso} d’envoyer un certain nombre de messages : facture en PDF, mot de passe de connexion, …
Vous pouvez préciser comment réagissent les liens “écrire à tous” réagis avec votre logiciel de messagerie.


\subsection{Les cadres associatifs}
\label{\detokenize{asso/first_step:les-cadres-associatifs}}\begin{quote}

\sphinxAtStartPar
Menu \sphinxstyleemphasis{Général/Nos coordonnées}

\sphinxAtStartPar
Menu \sphinxstyleemphasis{Administration/Modules (conf.)/Configuration des contacts}

\sphinxAtStartPar
Menu \sphinxstyleemphasis{Association/Adhérents/Adhérents cotisants}
\end{quote}

\sphinxAtStartPar
Dans la fenêtre de vos coordonnées, vous pouvez associer des adhérents comme cadres de votre structure.
Utilisez l’outil de recherche et assignez\sphinxhyphen{}leur une fonction.
Vous pouvez également rajouter des fonctions propres à votre structure.

\sphinxAtStartPar
Depuis la fiche de chacun de vos adhérents, vous pouvez donner à des personnes actives un droit de connexion à Diacamma Asso.
Privilégiez une utilisation du logiciel avec un alias et un mot de passe propre à chaque utilisateur. Associez\sphinxhyphen{}leur également les droits correspondant à leurs fonctions au sein de votre structure.
Enfin, évitez autant que possible l’utilisation de l’alias “admin” qui doit être réservé pour des actions de configuration et de maintenance.


\subsection{La gestion documentaire}
\label{\detokenize{asso/first_step:la-gestion-documentaire}}\begin{quote}

\sphinxAtStartPar
Menu \sphinxstyleemphasis{Administration/Modules (conf.)/Dossier}

\sphinxAtStartPar
Menu \sphinxstyleemphasis{Bureautique/Documents/Documents}
\end{quote}

\sphinxAtStartPar
Définissez vos différents dossiers vous permettant d’importer vos documents à classer et à partager.

\sphinxAtStartPar
Après avoir parcouru ces points, votre logiciel \sphinxstyleemphasis{Diacamma Asso} est pleinement opérationnel.
N’hésitez pas à consulter le forum : de nombreuses astuces peuvent vous aider à utiliser au mieux votre logiciel.

\sphinxstepscope


\chapter{Diacamma adhérent}
\label{\detokenize{member/index:diacamma-adherent}}\label{\detokenize{member/index::doc}}
\sphinxAtStartPar
Aide relative aux fonctionnalités de gestion d’adhésion.

\sphinxstepscope


\section{L’adhérent}
\label{\detokenize{member/member:ladherent}}\label{\detokenize{member/member::doc}}
\sphinxAtStartPar
L’adhérent est une personne physique qui a pris une inscription dans votre association.
Consultez la liste de vos adhérents :
\begin{quote}

\sphinxAtStartPar
Menu \sphinxstyleemphasis{Association/Adhérents/Adhérents cotisants}
\end{quote}

\sphinxAtStartPar
De cet écran vous pouvez, après avoir filtré vos adhérents sur un ou plusieurs critères, imprimer les listes, les étiquettes ou les cartes d’adhérents.
Sélectionnez et éditez une ligne pour consulter la fiche de l’adhérent.

\sphinxAtStartPar
\sphinxstyleemphasis{Diacamma Asso} sauvegarde plusieurs types d’informations relatives à un adhérent. Vous pouvez les consulter dans la fiche de l’adhérent.
\begin{description}
\sphinxlineitem{Cette fiche comporte :}\begin{itemize}
\item {} 
\sphinxAtStartPar
L’identité de l’adhérent : nom, prénom, adresse, téléphone, date et lieu de naissance…

\item {} 
\sphinxAtStartPar
La liste de ses adhésions

\item {} 
\sphinxAtStartPar
D’autres onglets dépendent des extensions installées : liste des diplômes/formations/grades ou situation comptable

\end{itemize}

\end{description}

\sphinxAtStartPar
Si vous avez défini une liste de documents d’inscription pour la saison, vous pouvez, de manière individuelle, cocher les documents rendus. Une impression de l’état de l’ensemble des adhérents en la matière est disponible depuis la liste des adhérents.
Etant donné qu’un adhérent est aussi une personne physique, vous retrouvez ces même fiches depuis la liste des contacts. Il vous est également possible de promouvoir un simple contact physique en adhérent ; vous aurez alors juste à compléter sa fiche d’adhérent.
Enfin, depuis la fiche d’un adhérent, il est possible de lui donner un droit d’accès, ou alias, à \sphinxstyleemphasis{Diacamma}. (voir Les utilisateurs).

\sphinxAtStartPar
En fonction de vos droits de connexion, il vous sera possible de modifier l’identité d’un adhérent, d’ajouter une adhésion passée ou un diplôme obtenu.

\sphinxAtStartPar
\sphinxstylestrong{Attention:} Le règlement général sur la protection des données (RGPD) n’autorise la conservation de données que si elles sont relatives à l’activité de nos adhérents. Veillez donc à bien adapter vos commentaires à ce cadre législatif et à informer toute personne de son inscription dans votre base de données.
Consultez la réglementation de loi en cas de doute, et n’hésitez pas à contacter la CNIL en cas de besoin (\sphinxurl{https://www.cnil.fr/fr/reglement-europeen-protection-donnees}).

\sphinxstepscope


\section{Rechercher un adhérent}
\label{\detokenize{member/member_search:rechercher-un-adherent}}\label{\detokenize{member/member_search::doc}}\begin{quote}

\sphinxAtStartPar
Menu \sphinxstyleemphasis{Association/Adhérents/Recherche d’adhérents}
\end{quote}

\noindent{\hspace*{\fill}\sphinxincludegraphics[height=400\sphinxpxdimen]{{search}.png}\hspace*{\fill}}

\sphinxAtStartPar
Vous pouvez alors faire une recherche suivant des critères variés portant sur l’identité ou les adhésions.
Vous pouvez également rechercher des adhérents sur des “documents demandés” ou sur un champ personnalisé.

\sphinxAtStartPar
\sphinxstylestrong{Remarque:} si vous faites une recherche avec des critères liés à une cotisation (équipe, documents, numéro licence, …) n’oubliez pas de préciser la saison de recherche. Le résultat donne une liste d’adhérents correspondant aux critères fournis.

\sphinxAtStartPar
Il est possible de fusionner plusieurs fiches d’une même personne.
Pour cela vous devez préciser la personne principale, l’outil supprimera les autres fiches après avoir déplacé toutes leurs références sur l’enregistrement principal.
Si vous voulez supprimer un adhérent, celui\sphinxhyphen{}ci ne devra pas avoir eu d’activité.

\sphinxAtStartPar
Depuis l’outil de recherche, vous pouvez aussi rechercher les fiches adhérents doublons. Cela vous permet d’afficher la liste des adhérents ayant les nom et prénom similaires.
De là, vous pouvez également fusionner ou supprimer la ou les fiches redondantes.

\sphinxstepscope


\section{Les cotisations}
\label{\detokenize{member/subscription:les-cotisations}}\label{\detokenize{member/subscription::doc}}
\sphinxAtStartPar
Ce qui caractèrise un adhérent par rapport à un contact physique, c’est le fait qu’il soit membre de votre association et qu’il soit à jour de cotisation.

\sphinxAtStartPar
Une cotisation symbolise l’acte d’adhésion à votre structure pendant un laps de temps défini.
Une personne sera donc considérée comme adhérente de votre association à une date donnée si elle a souscrit à une cotisation sur une période de temps comprenant cette date.


\subsection{Les états d’une cotisation}
\label{\detokenize{member/subscription:les-etats-d-une-cotisation}}
\sphinxAtStartPar
La cotisation d’un adhérent peut avoir plusieurs états :
\begin{quote}
\begin{itemize}
\item {} 
\sphinxAtStartPar
en attente :

\end{itemize}

\sphinxAtStartPar
État transitoire accessible seulement si vous souhaitez donner à vos adhérents la possibilité de demander en ligne une cotisation.
Cet état nécessite alors une modération pour la passer à \sphinxstyleemphasis{en création}.
\begin{itemize}
\item {} 
\sphinxAtStartPar
en création :

\end{itemize}

\sphinxAtStartPar
État correspondant à une préinscription.
La cotisation est alors associée à un devis pour approbation de la part de l’adhérent.
En cas de non acceptation, cette cotisation est \sphinxstyleemphasis{annulée} ou simplement supprimée.
\begin{itemize}
\item {} 
\sphinxAtStartPar
validée :

\end{itemize}

\sphinxAtStartPar
État général correspondant à un adhérent à part entière dans votre association.
Le devis est alors transformé en facture.
La cotisation est associée à une facture.
\begin{itemize}
\item {} 
\sphinxAtStartPar
annulée :

\end{itemize}

\sphinxAtStartPar
État permettant de suivre les anciens adhérants n’ayant pas souhaité confirmer leur adhésion.
\begin{itemize}
\item {} 
\sphinxAtStartPar
radiée :

\end{itemize}

\sphinxAtStartPar
État permettant de noter qu’un adhérent a été exclu de l’association.
À noter qu’en cas d’erreur, il est possible de rééditer cette cotisation pour la repasser à \sphinxstyleemphasis{en création}. Un nouveau devis de cotisation est alors régénéré.
\end{quote}

\sphinxstepscope


\section{Renouvellements}
\label{\detokenize{member/renew:renouvellements}}\label{\detokenize{member/renew::doc}}
\sphinxAtStartPar
Quand vous commencez une nouvelle saison, vous avez un certain nombre d’adhérents qui renouvellent leur adhésion. Si vous avez déjà enregistré cette personne dans votre logiciel, vous pouvez rechercher sa fiche et lui rajouter une licence pour la nouvelle saison.

\sphinxAtStartPar
Si vous avez beaucoup de renouvellements ce traitement serait très long !
\begin{quote}

\sphinxAtStartPar
Menu \sphinxstyleemphasis{Association/Adhérents/Adhérents à renouveler}
\end{quote}

\sphinxAtStartPar
Par défaut, cette écran à l’option « cotisations à rappeler » coché: cela vous permet de voir l’ensemble de vos adhérents dont l’adhésion est à « en création ».
Cela correspond à ceux qui sont en cours de renouvellement et qui n’ont pas encore régler leur cotisation.
Vous avez la possibilité d’éditer cette liste (bouton « Liste » et « Etiquette »).
Vous pouvez également envoyer une relance en selectionnant ces adhérents et en cliquant sur « Envoyer » : vous pouvez alors leur envoyer soit par courriel, soit en imprimant leur devis.

\sphinxAtStartPar
En décochant l’option « cotisations à rappeler », vous obtenez alors la liste des anciens adhérents, plus actif dans votre structure à l’intervalle de fin de cotisation de la date de référence (et qui n’ont pas de cotisation « en création » à la date de référence).
\begin{itemize}
\item {} 
\sphinxAtStartPar
Si l’intervalle est négatif : c’est des adhérents qui ont déjà terminé leur adhésion.

\item {} 
\sphinxAtStartPar
Si l’intervalle est positif : c’est des adhérents qui vont bientôt terminer leur adhésion.

\end{itemize}

\sphinxAtStartPar
Pour leur ajouter la nouvelle cotisation, deux solutions :
\begin{itemize}
\item {} \begin{description}
\sphinxlineitem{De manière unitaire :}
\sphinxAtStartPar
ouvrez chaque fiche et ajoutez la nouvelle licence

\end{description}

\item {} \begin{description}
\sphinxlineitem{De manière globale :}
\sphinxAtStartPar
sélectionnez plusieurs fiches dans la liste et cliquez sur « Renouveler ». Chaque personne sera alors renouvelée dans les mêmes conditions que la saison précédente (même cotisation, même N° de licence, même équipe, même activité)
Une cotisation « en création » est alors créé avec son devis associé validé.
Vous avez alors la possibilité d’envoyer ce devis par courriel ou l’imprimer via un PDF.

\end{description}

\end{itemize}

\sphinxstepscope


\section{Prestations de cotisation}
\label{\detokenize{member/prestation:prestations-de-cotisation}}\label{\detokenize{member/prestation::doc}}
\sphinxAtStartPar
Cela permet d’associer une équipe/cours à un article facturable afin de proposer un choix de prestations supplémentaires au moment de la saisie de la cotisation.

\sphinxAtStartPar
Vous choisissez vos prestations au moment de votre prise de cotisation ou via un écran de gestion spécifique.
Automatiquement, votre adhérent est alors associé à la bonne catégorie d’équipes/cours définie par la prestation.
De plus, dans la facture d’adhésion est ajouté alors l’article relatif à cette prestation.

\sphinxAtStartPar
Pour activer ce mode, vous devez configurer dans le menu \sphinxstyleemphasis{Administration/Modules (conf.)/Catégories}, le paramètre \sphinxstyleemphasis{Activer les équipes} = « Avec prestation »
Une fois alors votre Diacamma rafraichi, un nouvelle écran \sphinxstyleemphasis{Association/Liste de prestations}

\noindent{\hspace*{\fill}\sphinxincludegraphics[height=400\sphinxpxdimen]{{listprestation}.png}\hspace*{\fill}}


\subsection{Créer un prestation}
\label{\detokenize{member/prestation:creer-un-prestation}}
\sphinxAtStartPar
Depuis l’écran \sphinxstyleemphasis{Association/Liste de prestations}, il vous est possible de créer une nouvelle prestation.

\sphinxAtStartPar
Pour cela, cliquez sur le bouton « Créer ».

\noindent{\hspace*{\fill}\sphinxincludegraphics[height=400\sphinxpxdimen]{{newprestation}.png}\hspace*{\fill}}

\sphinxAtStartPar
Vous pouvez alors créer une prestation soit en créant une équipe/cours, soit en l’associant à une existante.
Si vous gerez des activités, vous devez alors associer cette prestation à une activité existante.
De plus, vous devez également définir un article de facturation à cette prestation.

\sphinxAtStartPar
Vous pouvez bien entendu, ensuite modifier cette prestation: nom et description de son équipe/cours, son activité et son article.

\sphinxAtStartPar
Si vous le voulez alors, vous pouvez également lui associé d’autres articles afin que celle\sphinxhyphen{}ci comporte plusieurs tarif.
Cochez « Utiliser le multi\sphinxhyphen{}prix » pour que l’interface vous propose l’ajout, la modification ou la suppression d’un article de prix.

\noindent{\hspace*{\fill}\sphinxincludegraphics[height=400\sphinxpxdimen]{{modifprestation}.png}\hspace*{\fill}}

\sphinxAtStartPar
Si vous souhaitez supprimer une prestation, il vous sera demandé ce que vous souhaitez faire de l’équipe/cours associée: la désactiver, la supprimer ou la laisser.


\subsection{Associer des pratiquants}
\label{\detokenize{member/prestation:associer-des-pratiquants}}
\sphinxAtStartPar
A chacune de ces prestations, vous pouvez bien sur ajouter des pratiquants.

\noindent{\hspace*{\fill}\sphinxincludegraphics[height=400\sphinxpxdimen]{{showprestation}.png}\hspace*{\fill}}

\sphinxAtStartPar
Via le bouton « Ajouter », rechercher une (ou plusieurs) fiche adhérent qui sera alors associé à cette prestation.
Vous pouvez de là également créer une nouvelle fiche.
Si cet adhérent n’est pas encore cotisant à votre structure, il vous sera demandé un type de cotisation.
Si la cotisation a plusieurs prix possible, il vous sera demandé quelle tarification utiliser.

\sphinxAtStartPar
Dans le cas où un adhérent a sa cotisation « en création », les articles de la prestations choisis serons alors automatiquement ajouter au devis associé à cette cotisation.
Si sa cotisation est « validée », une facture est alors généré afin de prendre en compte cette nouvelle prestation.
Dans le cas où l’adhérent est retiré d’une prestation, un avoir équivalent est alors créé.


\subsection{Gérer des prestations}
\label{\detokenize{member/prestation:gerer-des-prestations}}
\sphinxAtStartPar
Bien entendu, vos prestations ne sont pas figées.

\sphinxAtStartPar
Vous pouvez vouloir inverser des pratiquants, d’une prestation à une autre.
Pour cela, cliquez sur « Permuter » après avoir sélectionné 2 prestations: un écran vous invitera à associer les bons pratiquants au bon groupe.

\noindent{\hspace*{\fill}\sphinxincludegraphics[height=400\sphinxpxdimen]{{swapingprestation}.png}\hspace*{\fill}}

\sphinxAtStartPar
Dans la même idée, le bouton « Dédoubler » vous permet de créer une nouvelle prestation d’après une ancienne et d’ensuite pouvoir permuter les pratiquants comme vous le souhaitez.

\sphinxAtStartPar
Le bouton « Fusion » permet, quand à lui, de fusionner en une seule prestation celles que vous aurez sélectionnées.
Les équipes/cours non conservées sont alors supprimées.

\sphinxAtStartPar
Avec cette gestion, la facturation s’en retrouvera alors automatiquement impacté avec la création de devis, facture ou avoir, suivant les modifications apportées.

\sphinxstepscope


\section{Statistiques des adhérents}
\label{\detokenize{member/statistic:statistiques-des-adherents}}\label{\detokenize{member/statistic::doc}}\begin{quote}

\sphinxAtStartPar
Menu \sphinxstyleemphasis{Association/Adhérents/Statistiques}
\end{quote}

\noindent{\hspace*{\fill}\sphinxincludegraphics[height=400\sphinxpxdimen]{{statistiques1}.png}\hspace*{\fill}}

\sphinxAtStartPar
Vous avez la possibilité d’éditer les statistiques d’adhésion d’une saison donnée.

\sphinxAtStartPar
Ce document vous donne les effectifs par ville et par genre. Il est possible aussi de distinguer les différentes durées de cotisations.

\sphinxstepscope


\section{Reçus fiscaux}
\label{\detokenize{member/taxreceipt:recus-fiscaux}}\label{\detokenize{member/taxreceipt::doc}}
\sphinxAtStartPar
Si votre association est d’intérêt général, elle peut délivrer des reçus fiscaux à ses donateurs et à ses membres. Vous pouvez utiliser \sphinxstyleemphasis{Diacamma} pour cela.


\subsection{Configuration}
\label{\detokenize{member/taxreceipt:configuration}}
\sphinxAtStartPar
\sphinxstylestrong{Indiquer les codes comptables des produits pouvant donner droit à une déduction fiscale}
\begin{quote}

\sphinxAtStartPar
Menu \sphinxstyleemphasis{Administration/Configuration générale} \sphinxhyphen{} onglet « Adhérents »
\end{quote}

\sphinxAtStartPar
Mettez à jour le champ « code comptable des reçus fiscaux » avec les comptes de produits concernés.
Après rafraîchissement de l’application, le menu \sphinxstyleemphasis{Comptabilité/Reçus fiscaux} est maintenant disponible.

\sphinxAtStartPar
\sphinxstylestrong{Ajouter une image de signature}
\begin{quote}

\sphinxAtStartPar
Menu \sphinxstyleemphasis{Bureautique/Gestion de fichiers et de documents/Documents}
\end{quote}

\sphinxAtStartPar
Ajoutez dans le \sphinxstyleemphasis{Gestionnaire de documents} le fichier image devant être utilisé comme cachet ou signature de votre association.
\begin{quote}

\sphinxAtStartPar
Menu \sphinxstyleemphasis{Administration/Configuration générale} \sphinxhyphen{} onglet « Gestion documentaire »
\end{quote}

\sphinxAtStartPar
Dans un second temps, mettez à jour le champ « image de cachet ou de signature ».
Cette image, intégrée dans le patron d’impression par défaut, sera insérée automatiquement dans le reçu au format PDF, zone « signature ».

\sphinxAtStartPar
\sphinxstylestrong{Préciser la justification du reçu fiscal}

\sphinxAtStartPar
L’association doit indiquer sur les reçus à quel titre elle les émet ce reçu.
En général, elle précise la date de parution au Journal Officiel de sa situation d’utilité publique.
\begin{quote}

\sphinxAtStartPar
Menu \sphinxstyleemphasis{Général/Nos coordonnées}
\end{quote}

\sphinxAtStartPar
Éditez la fiche de votre association et renseignez le champ « N° SIRET/SIREN ». Ce champ est ajouté par défaut en pied de page du reçu.

\sphinxAtStartPar
\sphinxstylestrong{Ajouter le logo de votre structure}

\sphinxAtStartPar
Afin d’éviter d’avoir le logo par défaut dans vos reçus, enregistrez sous \sphinxstyleemphasis{Diacamma} le logo de votre association.
\begin{quote}

\sphinxAtStartPar
Menu \sphinxstyleemphasis{Général/Nos coordonnées}
\end{quote}

\sphinxAtStartPar
Éditez la fiche de votre association et associez\sphinxhyphen{}lui une image.


\subsection{Génération des reçus}
\label{\detokenize{member/taxreceipt:generation-des-recus}}\begin{quote}

\sphinxAtStartPar
Menu \sphinxstyleemphasis{Comptabilité/Reçus fiscaux}
\end{quote}

\sphinxAtStartPar
Visualisez les reçus fiscaux déjà produits et filtrez\sphinxhyphen{}les sur l’année civile.
L’ouverture du droit à déduction fiscale nait du règlement du produit ouvrant droit à déduction. C’est pour cela que le reçu fiscal est établi sur l’année du règlement. En cas de règlement fractionné, c’est le dernier versement qui fait naître le droit à déduction.

\noindent{\hspace*{\fill}\sphinxincludegraphics[height=400\sphinxpxdimen]{{taxreceipts}.png}\hspace*{\fill}}

\sphinxAtStartPar
Les reçus fiscaux de l’année sélectionnée peuvent être générés, via le bouton « Contrôle » (situé en bas à droite) :
\begin{itemize}
\item {} 
\sphinxAtStartPar
L’outil extrait tous les mouvements satisfaisant aux conditions suivantes :
\begin{itemize}
\item {} 
\sphinxAtStartPar
les codes comptables doivent correspondre à des produits ouvrant droit à déduction fiscale (Menu \sphinxstyleemphasis{/Configuration générale})

\item {} 
\sphinxAtStartPar
les écritures associées doivent avoir été validées

\item {} 
\sphinxAtStartPar
les créances liées aux mouvements doivent être lettrées afin d’attester qu’elles sont bien réglées

\item {} 
\sphinxAtStartPar
le règlement doit avoir eu lieu sur l’année spécifiée

\item {} 
\sphinxAtStartPar
les mouvements ne doivent pas être déjà associés à des reçus fiscaux

\end{itemize}

\end{itemize}

\sphinxAtStartPar
Les mouvements satisfaisant à ces cinq conditions sont reportés dans le reçu de chaque tiers concerné.
\begin{quote}
\begin{itemize}
\item {} 
\sphinxAtStartPar
Attribution d’un numéro unique (spécifique à l’année civile) à tout reçu fiscal

\item {} 
\sphinxAtStartPar
Possibilité de gérer des reçus fiscaux pour « abandon de frais »

\end{itemize}

\sphinxAtStartPar
Ces dons peuvent prendre plusieurs formes (argent, abandon de revenus ou de produits, renonciation expresse à des frais engagés dans le cadre d’une activité bénévole respectant certaines conditions)
\end{quote}

\sphinxAtStartPar
Une fois généré, chaque reçu pourra être imprimé en PDF en mode « simulation », afin de pouvoir vérifier le résultat.
Depuis la fiche d’un reçu, il est possible de relancer la génération via le bouton « Contrôle ».

\sphinxAtStartPar
\sphinxstylestrong{Remarque :} Si vous avez défini des rubriques de compte de résultat (voir le manuel « Diacamma Comptabilité », « Création, modification et édition de comptes du plan comptable »),
ceux\sphinxhyphen{}ci sont alors utilisé dans le reçu fiscal en précision de nature du don.

\sphinxAtStartPar
Après les avoir bien vérifiés, cliquer sur le bouton « Validées » afin de figer ces différents reçu.
Il peut alors être envoyé par courriel, en PDF (tout comme les factures).

\sphinxAtStartPar
\sphinxstylestrong{Attention :} la validation de reçus fiscaux est définitive.
Comme ils correspondent à la réalité d’une comptabilité validée, il n’est pas possible de les corriger ou de les annuler.

\sphinxAtStartPar
\sphinxstylestrong{Note:} Une version PDF des reçus validés est automatiquement sauvegardée dans le \sphinxstyleemphasis{Gestionnaire de documents}, dans un répertoire ayant pour nom l’année civile.
Si ce répertoire n’existe pas, il est créé.

\sphinxstepscope


\section{Configuration}
\label{\detokenize{member/config:configuration}}\label{\detokenize{member/config::doc}}
\sphinxAtStartPar
Pour vous permettre d’exploiter au mieux le logiciel, certains paramétrages sont nécessaires.


\subsection{Générale}
\label{\detokenize{member/config:generale}}\begin{quote}

\sphinxAtStartPar
Menu \sphinxstyleemphasis{Administration/Configuration générale}, onglet \sphinxstyleemphasis{Adhérents}
\end{quote}

\sphinxAtStartPar
Vous pouvez modifier les paramètrages généraux relatifs à la gestion des adhésions.

\noindent{\hspace*{\fill}\sphinxincludegraphics[height=400\sphinxpxdimen]{{conf_general}.png}\hspace*{\fill}}
\begin{itemize}
\item {} 
\sphinxAtStartPar
\sphinxstylestrong{Type de structure “famille”} : Pour activer la gestion des adhérents par famille.
Vous devez préciser ici comment les familles  seront classifiées(structure morales).

\item {} 
\sphinxAtStartPar
\sphinxstylestrong{Activer la création de connexion} : Permet d’attribuer un accès de connexion par adhérent.
« Par adhérent »: une action permettra d’attribué une connexion à tout les adhérents actifs.
« Après demande »: l’outil « mot de passe oublié » permet d’activé une connexion si l’adhérents associé est actif.

\item {} 
\sphinxAtStartPar
\sphinxstylestrong{Mode de cotisation} : Possibilité de créer une cotisation à la demande.

\item {} 
\sphinxAtStartPar
\sphinxstylestrong{Résultats par page} : Taille par défaut de la pagination des adhérents.

\item {} 
\sphinxAtStartPar
\sphinxstylestrong{Catégorie de facturation pour les cotisations} : Catégorie à appliquer pour la facturation des cotisations.

\item {} 
\sphinxAtStartPar
\sphinxstylestrong{code comptable des reçus fiscaux} : Préciser la liste des codes comptable soumis à défiscalisation.

\item {} 
\sphinxAtStartPar
\sphinxstylestrong{Règlement par défaut des reçus fiscaux} : Pour les règlements soumis à défiscalisation et qui n’aurait pas de mode défini.

\item {} 
\sphinxAtStartPar
\sphinxstylestrong{groupe des adhérents actifs} : Groupe de connexion associé à un adhérent actif (enlevé si non renouvelé).

\item {} 
\sphinxAtStartPar
\sphinxstylestrong{Filtrage pour le renouvellement} : Filtrage supplémentaire pour l’écran de renouvellement d’adhésion.

\end{itemize}


\subsection{Saisons et cotisations}
\label{\detokenize{member/config:saisons-et-cotisations}}\begin{quote}

\sphinxAtStartPar
Menu \sphinxstyleemphasis{Administration/Modules (conf.)/Les saisons et les cotisations}
\end{quote}

\sphinxAtStartPar
Vous pouvez modifier les paramètres de la gestion des licences sportives, en l’occurrence, les saisons sportives et les types de cotisations.

\sphinxAtStartPar
\sphinxstylestrong{Les Saisons}

\noindent{\hspace*{\fill}\sphinxincludegraphics[height=400\sphinxpxdimen]{{season_list}.png}\hspace*{\fill}}

\sphinxAtStartPar
Ici, vous pourrez ajouter de nouvelles saisons au fur et à mesure de l’utilisation du logiciel. Vous pourrez également déterminer la saison courante (dite active).
De plus, chaque saison est découpée en quatre périodes et en douze mois.

\sphinxAtStartPar
C’est la plus petite date de début et la plus grande de fin qui définissent la plage de votre saison. Vos saisons peuvent être d’une durée supérieure à un an et peuvent se chevaucher.
Vous pouvez modifier les dates de début et de fin de chaque période. Vous pouvez aussi ajouter ou supprimer une période, mais vous devez toujours en avoir au moins deux.
Deux périodes peuvent se chevaucher ou être disjointes (dans ce cas, un message d’avertissement vous prévient).
Le premier des douze mois commence le mois de la plus petite date de début de période. Même si votre saison couvre plus d’une année calendaire, il n’y aura pas de treizième mois dans votre saison.

\sphinxAtStartPar
Vous pouvez associer à chaque saison une liste de documents que chaque adhérent devra vous fournir pour finaliser son inscription.

\noindent{\hspace*{\fill}\sphinxincludegraphics[height=400\sphinxpxdimen]{{documents}.png}\hspace*{\fill}}

\sphinxAtStartPar
\sphinxstylestrong{Les types de cotisation}

\noindent{\hspace*{\fill}\sphinxincludegraphics[height=400\sphinxpxdimen]{{cotisations}.png}\hspace*{\fill}}

\sphinxAtStartPar
Ici vous pourrez saisir les différents types de cotisation proposés par votre association. Par exemple, pour une association pratiquant plusieurs activités sportives distinctes, vous pouvez avoir un type de cotisation pour chaque activité, un autre pour plusieurs de ces activités, et encore des types différents selon une pratique en compétition ou hors compétition des activités.

\sphinxAtStartPar
Quatre modes de durées différentes peuvent être affectées à un type de cotisation :
\begin{itemize}
\item {} 
\sphinxAtStartPar
Annuelle : cotisation couvrant l’ensemble de la saison.

\item {} 
\sphinxAtStartPar
Périodique : cotisation couvrant une période (4 par défaut) de la saison.

\item {} 
\sphinxAtStartPar
Mensuelle : cotisation couvrant un des douze mois de la saison.

\item {} 
\sphinxAtStartPar
Calendaire : cotisation couvrant une année calendaire. Cette cotisation peut donc être à cheval sur deux saisons.

\end{itemize}

\sphinxAtStartPar
Pour le lien avec le module \sphinxstyleemphasis{Facturier} vous devez définir un prix de vente de votre cotisation en créant et en associant des articles.
Vous pouvez rattacher plusieurs articles à  une cotisation. Cela vous permet de distinguer, par exemple, la part de cotisation relative à votre club, de la licence de votre fédération.
Avec ces liens entre les cotisations et les articles,vous pourrez générer automatiquement des factures lors de vos procédures d’adhésion. Si une cotisation n’est liée au aucun article, aucune facture ne sera émise.

\sphinxAtStartPar
De même, vous pouvez aussi personnaliser le code comptable du tiers associé à vos adhérants dans le cas d’un création automatique.


\subsection{Catégories}
\label{\detokenize{member/config:categories}}
\sphinxAtStartPar
Le menu \sphinxstyleemphasis{Administration/Modules (conf.)/Catégories} vous permet de modifier ce qui peut catégoriser un adhérent : les catégories d’âge, les équipes ou cours et les activités ainsi que la possibilité d’activer ou non ces différentes classifications.

\sphinxAtStartPar
Vous pouvez ne pas vouloir utiliser certaines catégories. Pour cela, désactivez\sphinxhyphen{}les depuis l’écran de paramétrage.
De la même façon, vous pouvez préciser si vous souhaitez pouvoir créer automatiquement une connexion par adhérent actif, afficher un numéro d’adhérent ou gérer des numéros de licence.
Vous pouvez également personnaliser la désignation “équipe” et “activité”.

\noindent{\hspace*{\fill}\sphinxincludegraphics[height=400\sphinxpxdimen]{{categories}.png}\hspace*{\fill}}

\sphinxAtStartPar
\sphinxstylestrong{Les âges}

\sphinxAtStartPar
Vous pourrez ici renseigner les catégories d’âges existantes dans votre association avec un nom de catégorie, une année (de naissance) de début et de fin de la catégorie.

\sphinxAtStartPar
Vous n’aurez pas besoin de changer les valeurs des années de naissance ultérieurement : le décalage est effectué automatiquement d’année en année.

\noindent{\hspace*{\fill}\sphinxincludegraphics[height=400\sphinxpxdimen]{{age}.png}\hspace*{\fill}}

\sphinxAtStartPar
\sphinxstylestrong{Les équipes/cours}

\sphinxAtStartPar
Vous gérez différentes équipes ou différents cours et vous souhaitez pouvoir gérer vos adhérents selon ce critère.
Renseignez\sphinxhyphen{}les ici, vous pourrez alors affecter des adhérents à ces équipes ou cours et ainsi les retrouver plus facilement.

\noindent{\hspace*{\fill}\sphinxincludegraphics[height=400\sphinxpxdimen]{{team}.png}\hspace*{\fill}}

\sphinxAtStartPar
\sphinxstylestrong{Les activités}

\sphinxAtStartPar
Vous gérez différentes activités (par exemple plusieurs arts martiaux) dans votre association ? Les renseigner ici vous permettra ensuite de classer vos adhérents en fonction de ces différentes activités, mais aussi de saisir pour eux plusieurs licences par an si nécessaire.

\sphinxAtStartPar
Exemple : une association regroupant judo et karaté, et donc affiliée à deux fédérations sportives différentes.
Vous pourriez alors saisir 2 licences par adhérent (sous réserve que vos adhérents pratiquent les deux sports et soient licenciés des deux fédérations).

\noindent{\hspace*{\fill}\sphinxincludegraphics[height=400\sphinxpxdimen]{{activity}.png}\hspace*{\fill}}

\sphinxAtStartPar
Il est possible de supprimer ou de désactiver une activité.
Il n’est pas possible de supprimer une activité utilisée (par exemple dans une ancienne cotisation), dans ce cas là, désactivez la si vous ne vous en servez plus.
Il doit toujours avoir au moins une activité active.

\sphinxstepscope


\chapter{Diacamma évenement}
\label{\detokenize{event/index:diacamma-evenement}}\label{\detokenize{event/index::doc}}
\sphinxAtStartPar
Aide relative aux fonctionnalités de gestion d’évenement.

\sphinxstepscope


\section{Création d’un évenement}
\label{\detokenize{event/newevent:creation-d-un-evenement}}\label{\detokenize{event/newevent::doc}}
\sphinxAtStartPar
Un évenement, pour \sphinxstyleemphasis{Diacamma}, correspond à la gestion d’une activité particulière de votre structure.
\begin{description}
\sphinxlineitem{Elle peut être de deux types:}\begin{itemize}
\item {} 
\sphinxAtStartPar
Un passage d’examen
Pour vous aider à gérer un examen au sein de votre association afin de valider pour vos pratiquants un grade, un diplôme ou un niveau.
Il est nécessaire pour cela de configurer la liste des grades/niveaux relatifs à votre structure.

\item {} 
\sphinxAtStartPar
Un stage ou une sortie
Ce type d’évenement va correspond à un grand nombre de cas d’activité dans lequel vous souhaitez gérer une inscription.

\end{itemize}

\end{description}

\sphinxAtStartPar
La création d’un évenement est assez simple.
Commencez simplement à préciser le type, la description et les dates de celui\sphinxhyphen{}ci.

\sphinxAtStartPar
Assignez alors à cet évenement un équipe d’organisation (dont un responsable) ainsi que des participants.
Les participants peuvent aussi bien être adhérent ou de simple contact: une colonne permet rapidement de connaitre leur statut.

\sphinxAtStartPar
A chaque participant, vous pouvez assigner un article de facture (un article par défaut simplifie également cette gestion).
Une fois l’évenement validé, une facture est ainsi générée en lien avec l’article du participant.

\sphinxAtStartPar
Dans le cas d’un passage d’examen, à la validation, vous devez également préciser le résultat de chacun des participants.
Tout les participants étant un adhérent de la structure auront une modification de leur fiche afin de faire figurer, le cas échant, leur nouveau grade, diplôme ou niveau.

\sphinxstepscope


\section{Statistique de formation}
\label{\detokenize{event/statistic:statistique-de-formation}}\label{\detokenize{event/statistic::doc}}
\sphinxAtStartPar
Un outil statistique simple vous permet de d’avoir rapidement un résumé pour une saison donnée du nombre de diplômes attribué pour cette periode de temps.

\sphinxstepscope


\section{Configurations}
\label{\detokenize{event/config:configurations}}\label{\detokenize{event/config::doc}}
\sphinxAtStartPar
La configuration du gestionnaire d’évenement est seulement relative à la configuration des grades, diplômes ou niveaux.

\sphinxAtStartPar
Dans le menu \sphinxstyleemphasis{Administration/Configuration des diplômes} vous pouvez préciser la dénomination de vos diplômes
ainsi que saisir hierachiquement les diplômes et éventuellement les sous\sphinxhyphen{}diplômes relatifs à votre structures.

\sphinxstepscope


\chapter{Diacamma comptabilité}
\label{\detokenize{accounting/index:diacamma-comptabilite}}\label{\detokenize{accounting/index::doc}}
\sphinxAtStartPar
Aide relative aux fonctionnalités comptables.

\sphinxstepscope


\section{Définitions}
\label{\detokenize{accounting/definition:definitions}}\label{\detokenize{accounting/definition::doc}}\begin{quote}

\sphinxAtStartPar
\sphinxstylestrong{Remarques:} Ce module comptable est proche d’une comptabilité type « entreprise », néanmoins il ne respecte pas certaines exigences légales et fiscales en la matière.
Ce module ne peut être utilisé que pour la tenue des comptes de structures gérées par des bénévoles. Il n’est pas possible de l’utiliser pour la tenue des comptes de structures commerciales, concurrentielles ou professionnelles.
Le représentant légale de la structure utilisant ce module doit s’assurer que sa comptabilité respecte alors la législation en vigueur de son pays.
\end{quote}


\subsection{Exercice comptable}
\label{\detokenize{accounting/definition:exercice-comptable}}
\sphinxAtStartPar
L’exercice comptable est la période qui sépare deux déterminations du résultat. Il commence par une ouverture des comptes et se termine par une clôture de ceux\sphinxhyphen{}ci.

\sphinxAtStartPar
Pour ce qui est de la durée, le responsable de la structure doit respecter la réglementation. Il doit aussi tenir compte des statuts (association) et du règlement de copropriété (syndic), tant pour ce qui est du premier exercice que des suivants. Pour ce qui est de la modification de la durée des exercices, là encore il doit se référer aux textes réglementaires et à ceux qui régissent la vie de la structure.


\subsection{Tiers comptable}
\label{\detokenize{accounting/definition:tiers-comptable}}
\sphinxAtStartPar
Un tiers comptable est une personne physique ou morale avec laquelle une entité va avoir des échanges monétaires ou matériels (clients, fournisseurs, salariés, administrations…).


\subsection{Journaux comptables}
\label{\detokenize{accounting/definition:journaux-comptables}}
\sphinxAtStartPar
Un journal comptable est un regroupement d’écritures comptables permettant de classer celles\sphinxhyphen{}ci par nature.

\sphinxAtStartPar
Des journaux par défaut vous sont proposés.

\sphinxAtStartPar
Deux d’entre eux ne doivent pas être supprimés car utilisés lors de la clôture de l’exercice comptable et l’ouverture des comptes :
\begin{itemize}
\item {} 
\sphinxAtStartPar
journal des opérations diverses : écriture de détermination du résultat de l’exercice (en ce qui concerne une copropriété, ce résultat est en principe nul)

\item {} 
\sphinxAtStartPar
journal « Report à\sphinxhyphen{}nouveaux » : reprise des soldes des comptes de bilan en début d’exercice comptable

\end{itemize}

\sphinxAtStartPar
En ce qui concerne les autres journaux, vous pouvez adapter la liste type en créant, modifiant ou supprimant ceux qui vous sont proposés.


\subsection{Ecritures comptables}
\label{\detokenize{accounting/definition:ecritures-comptables}}
\sphinxAtStartPar
Une écriture comptable est la traduction dans les comptes d’une opération réalisée par ou pour le compte de votre structure et ayant une incidence sur son patrimoine. En respect du principe de la partie double, il y a égalité débits = crédits.

\sphinxAtStartPar
Par exemple, une écriture d’achat se schématise ainsi :
\begin{itemize}
\item {} 
\sphinxAtStartPar
une ou plusieurs lignes au débit des comptes de charges correspondant aux biens acquis ou aux services reçus (matériel, prime d’assurance…)

\item {} 
\sphinxAtStartPar
une ligne au crédit du compte tiers fournisseur pour le net\sphinxhyphen{}à\sphinxhyphen{}payer de la facture

\end{itemize}

\sphinxAtStartPar
Pour ce qui est des comptes de tiers, vous pouvez aussi sous \sphinxstyleemphasis{Diacamma} lettrer les lignes d’écriture.

\sphinxAtStartPar
Ex: la dette née d’un achat avec son règlement.


\subsection{Plan comptable de l’exercice}
\label{\detokenize{accounting/definition:plan-comptable-de-l-exercice}}
\sphinxAtStartPar
Le plan comptable de l’exercice est l’ensemble des comptes ouverts au titre de l’exercice comptable. Il doit respecter  la nomenclature officielle des comptes applicable dans votre pays.
\begin{description}
\sphinxlineitem{Sous {\color{red}\bfseries{}*}Diacamma », le plan comptable est spécifique à chaque exercice comptable. Vous pourrez le créer :}\begin{itemize}
\item {} 
\sphinxAtStartPar
soit  à partir du plan de comptes type mis à votre disposition par le logiciel

\item {} 
\sphinxAtStartPar
soit en recopiant le plan de comptes de l’exercice précédent

\end{itemize}

\end{description}

\sphinxAtStartPar
La création des comptes n’est pas totalement libre. Vous pouvez subdiviser un compte. Par contre, vous ne devez pas créer un compte hors de la nomenclature officielle.

\sphinxstepscope


\section{Exercice comptable}
\label{\detokenize{accounting/fiscalyear:exercice-comptable}}\label{\detokenize{accounting/fiscalyear::doc}}

\subsection{Paramétrages}
\label{\detokenize{accounting/fiscalyear:parametrages}}\begin{quote}

\sphinxAtStartPar
Menu \sphinxstyleemphasis{Administration/Modules (conf.)/Configuration comptable}
\end{quote}

\sphinxAtStartPar
Ouvrez l’onglet « Paramètres » et éditez\sphinxhyphen{}les avec le bouton « Modifier ». Paramétrez la devise et sa précision, la taille des codes comptables. Précisez aussi si vous avez l’intention ou non de mettre en place une comptabilité analytique.

\noindent{\hspace*{\fill}\sphinxincludegraphics[height=300\sphinxpxdimen]{{parameters}.png}\hspace*{\fill}}
\begin{itemize}
\item {} 
\sphinxAtStartPar
\sphinxstylestrong{Devise (iso)} : Devise utilisée dans la comptabilité (au format international

\item {} 
\sphinxAtStartPar
\sphinxstylestrong{Devise (précision)} : Nombre de chiffre après la virgule de la devise

\item {} 
\sphinxAtStartPar
\sphinxstylestrong{Taille des codes comptables} : Taille minimum des code comptables, des zéro seront ajouté en plus éventuellement.

\item {} 
\sphinxAtStartPar
\sphinxstylestrong{Comptabilité analytique obligatoire} : Rend obligatoire l’utilisation de code analytique sur les lignes d’écritures de produit ou de charge.

\item {} 
\sphinxAtStartPar
\sphinxstylestrong{Code comptable (filtre de rapport)} : Début de code comptable utiliser pour prédéfinir des filtrages dans des rapports.

\item {} 
\sphinxAtStartPar
\sphinxstylestrong{Date du jour pour une nouvelle écriture} : Si non coché, rend la date d’écriture non rempli, obligeant à la saisir.

\end{itemize}


\subsection{Création d’un exercice comptable}
\label{\detokenize{accounting/fiscalyear:creation-d-un-exercice-comptable}}
\sphinxAtStartPar
Lors de la mise en place de votre comptabilité sous \sphinxstyleemphasis{Diacamma Syndic}, vous aurez à spécifier le système comptable qui sera utilisé (ex. Plan comptable général français). \sphinxstylestrong{Attention :} une fois choisie, cette option ne sera plus modifiable.
\begin{quote}

\sphinxAtStartPar
Menu \sphinxstyleemphasis{Administration/Modules (conf.)/Configuration comptable} \sphinxhyphen{} Onglet « Liste d’exercices »
\end{quote}

\noindent{\hspace*{\fill}\sphinxincludegraphics[height=400\sphinxpxdimen]{{fiscalyear_list}.png}\hspace*{\fill}}

\sphinxAtStartPar
Vérifiez que le système comptable a été choisi et cliquez sur « + Créer » afin de renseigner les bornes du nouvel exercice.

\noindent{\hspace*{\fill}\sphinxincludegraphics[height=200\sphinxpxdimen]{{fiscalyear_create}.png}\hspace*{\fill}}

\sphinxAtStartPar
Pour le premier exercice sous \sphinxstyleemphasis{Diacamma Syndic}, saisissez la date de début et la date de fin de l’exercice puis cliquez sur le bouton « OK ». Les exercices suivants ont comme date de début le lendemain de la date de clôture de l’exercice précédent et seule la date de fin est à saisir.

\sphinxAtStartPar
Notez que le logiciel associe à chaque exercice un répertoire de stockage du \sphinxstyleemphasis{Gestionnaire de documents} : certains documents
officiels seront sauvegardés dans celui\sphinxhyphen{}ci. Le bouton « Contrôle » vous permet à tout moment de  vérifier que vos documents officiels ont bien été générés.

\sphinxAtStartPar
Votre nouvel exercice figure maintenant dans la liste des exercices. Il est \sphinxstylestrong{{[}en création{]}}. Lorsque plusieurs exercices ont été créés, vous devez activer celui sur lequel vous souhaitez travailler par défaut, à l’aide du bouton « Activé ».
\begin{quote}

\sphinxAtStartPar
\sphinxstyleemphasis{Onglet « Journaux » et Onglet « Champ personnalisé des tiers »}
\end{quote}

\sphinxAtStartPar
Depuis ce même écran de configuration, vous pouvez également modifier ou ajouter des journaux. Des champs personnalisés peuvent aussi être ajoutés à la fiche modèle de tiers comptable. Ceci peut être intéressant si vous voulez réaliser des recherches/filtrages sur des informations propres à votre fonctionnement.

\sphinxAtStartPar
Maintenant, vous devez fermer la fenêtre « Configuration comptable » et créer le plan comptable de votre structure.
\begin{quote}

\sphinxAtStartPar
Menu \sphinxstyleemphasis{Comptabilité/Gestion comptable/Plan comptable}
\end{quote}

\noindent{\hspace*{\fill}\sphinxincludegraphics[height=400\sphinxpxdimen]{{account_list}.png}\hspace*{\fill}}

\sphinxAtStartPar
Avec le bouton « + Initiaux », générez automatiquement votre propre plan comptable général à partir du plan de comptes type fourni par le logiciel.
Adaptez celui\sphinxhyphen{}ci aux besoins de votre structure avec les boutons « Ajouter » et « Supprimer ».
\begin{itemize}
\item {} \begin{description}
\sphinxlineitem{\sphinxstylestrong{Première tenue de comptabilité sous Diacamma}}{[}vous migrez sous Diacamma et avez des à\sphinxhyphen{}nouveaux à saisir.{]}
\sphinxAtStartPar
Avec le bouton « + Initiaux », générez automatiquement votre propre plan comptable général à partir du plan de comptes type.
Adaptez celui\sphinxhyphen{}ci aux besoins de votre structure avec les boutons « Ajouter » et « Supprimer ».
Quittez l’écran \sphinxstyleemphasis{Plan comptable} et ouvrez le menu \sphinxstyleemphasis{Comptabilité/Gestion comptable/Ecritures comptables}.
Saisissez vos soldes à\sphinxhyphen{}nouveaux en une seule écriture, en prenant bien soin de la contrôler, dans le journal « Report à nouveau ».
Ceci fait, réouvrez le menu \sphinxstyleemphasis{Comptabilité/Gestion comptable/Plan comptable}.

\end{description}

\item {} \begin{description}
\sphinxlineitem{\sphinxstylestrong{Exercice comptable suivant}}{[}Il ne s’agit pas de votre premier exercice sous \sphinxstyleemphasis{Diacamma}.{]}
\sphinxAtStartPar
Si ce n’est pas déjà réalisé, avec le bouton « Importer » (et non « Initiaux »), vous devez importer le plan comptable de l’exercice précédent.
Contrôlez l’importation et mettez à jour, si besoin, le plan comptable de l’exercice.
Suite à votre dernière assemblée générale, les excédents n\sphinxhyphen{}1 doivent être normalement ventilés avant clôture de l’exercice n\sphinxhyphen{}1. Si ce n’est pas le cas, vous devez passer cette écriture.
Clôturez l’exercice n\sphinxhyphen{}1. Son état est maintenant \sphinxstylestrong{{[}terminé{]}}. Le nouvel exercice est toujours \sphinxstylestrong{{[}en création{]}}.
Utilisez le bouton « Report à nouveau » afin que les soldes n\sphinxhyphen{}1 des comptes d’actif et de passif soient repris dans la comptabilité du nouvel exercice. Vous pouvez constater à l’écran que les soldes des comptes de bilan non soldés fin n\sphinxhyphen{}1 ont été repris. L’écriture correspondante  au journal « Report à nouveau » est générée et validée automatiquement.

\end{description}

\end{itemize}

\sphinxAtStartPar
\sphinxstylestrong{Afin d’achever l’ouverture de votre nouvel exercice, vous devez maintenant cliquez sur « Commencer ».}
\sphinxstylestrong{Votre exercice est maintenant {[}en cours{]}}.


\subsection{Création, modification et édition de comptes du plan comptable}
\label{\detokenize{accounting/fiscalyear:creation-modification-et-edition-de-comptes-du-plan-comptable}}\begin{quote}

\sphinxAtStartPar
Menu \sphinxstyleemphasis{Comptabilité/Gestion comptable/Plan comptable}
\end{quote}

\sphinxAtStartPar
A tout moment vous pouvez ajouter un nouveau compte dans votre plan comptable.

\noindent{\hspace*{\fill}\sphinxincludegraphics[height=200\sphinxpxdimen]{{account_new}.png}\hspace*{\fill}}

\sphinxAtStartPar
Référez\sphinxhyphen{}vous à la règlementation de votre pays. Votre structure peut avoir l’obligation de respecter certaines obligations pour ce qui est de leur plan comptable.

\sphinxAtStartPar
Un compte peut être modifié, tant pour ce qui est de son numéro que de son intitulé. Les imputations (lignes d’écritures) qui lui sont associées seront automatiquement modifiées. Le changement n’est permis que si le nouveau compte relève de la même catégorie comptable (charge, produit…).

\sphinxAtStartPar
Lorsque vous consultez un compte (bouton « Editer » ou double\sphinxhyphen{}clic), les écritures associées au compte sont affichées.

\noindent{\hspace*{\fill}\sphinxincludegraphics[height=400\sphinxpxdimen]{{account_edit}.png}\hspace*{\fill}}

\sphinxAtStartPar
Il vous est aussi possible de supprimer un compte du plan comptable à la condition qu’aucune écriture ne lui soit associée.

\sphinxAtStartPar
\sphinxstylestrong{Remarque :} Pour les comptes de charges et de produits, il vous est possible de définir une rubrique.
Cette rubrique sera utilisé pour ordonnancer votre compte de résultat.


\subsection{Clôture d’un exercice}
\label{\detokenize{accounting/fiscalyear:cloture-d-un-exercice}}
\sphinxAtStartPar
En fin d’exercice comptable, celui\sphinxhyphen{}ci est clôturé. Cette opération, définitive, se réalise sous le contrôle de votre
vérificateur aux comptes.

\sphinxAtStartPar
Au préalable, vous devez :
\begin{itemize}
\item {} 
\sphinxAtStartPar
Passer vos écritures d’inventaire (charges à payer, produits à recevoir, créances douteuses…)

\item {} 
\sphinxAtStartPar
Contrôler que toutes les charges et les produits ont bien été imputés en comptabilité analytique

\item {} 
\sphinxAtStartPar
Vérifier que vos dépenses et vos recettes sont bien ventilées sur vos différentes catégories

\item {} 
\sphinxAtStartPar
Vérifier que toutes vos dépenses ont été ventilées sur les copropriétaires, pour ce qui est des copropriétés

\item {} 
\sphinxAtStartPar
Affecter vos excédents conformément aux délibérations de votre assemblée générale

\item {} 
\sphinxAtStartPar
Valider les écritures provisoires au brouillard

\item {} 
\sphinxAtStartPar
Lettrer les comptes de tiers

\item {} 
\sphinxAtStartPar
Créer l’exercice suivant si cela n’a pas été réalisé

\item {} 
\sphinxAtStartPar
Sauvegarder votre dossier
\begin{quote}

\sphinxAtStartPar
Menu \sphinxstyleemphasis{Comptabilité/Gestion comptable/Plan comptable}
\end{quote}

\end{itemize}

\sphinxAtStartPar
Cliquez sur le bouton « Clôturer ».

\sphinxAtStartPar
La clôture a pour effet de :
\begin{itemize}
\item {} 
\sphinxAtStartPar
Solder les comptes de gestion

\item {} 
\sphinxAtStartPar
Interdire tout ajout d’écriture

\item {} 
\sphinxAtStartPar
Arrêter les comptes de bilan et les comptes de tiers (copropriétaires, fournisseurs…)

\item {} 
\sphinxAtStartPar
Assurer qu’il ne pourra plus être apporté de modification à l’exercice clôturé

\end{itemize}

\sphinxAtStartPar
\sphinxstylestrong{Remarques :}
\begin{itemize}
\item {} 
\sphinxAtStartPar
Tant qu’un exercice n’est pas clôturé, vous pouvez enregistrer des opérations sur celui\sphinxhyphen{}ci et le suivant

\item {} 
\sphinxAtStartPar
Certaines structures ont des règles de clôture spécifique (exemple les ASL): bien verifier votre règlementation comptable en la matière.

\end{itemize}

\sphinxstepscope


\section{Tiers comptable}
\label{\detokenize{accounting/third:tiers-comptable}}\label{\detokenize{accounting/third::doc}}
\sphinxAtStartPar
\sphinxstylestrong{Préambule} :
Les opérations réalisées par votre structure avec ses partenaires donnent lieu à des échanges. Les flux physiques (échanges de biens et de services) et les flux monétaires (échanges de monnaie) doivent être constatés en comptabilité.

\sphinxAtStartPar
Pour chacun de vos partenaires, que ce soit des personnes physiques (individus) ou des personnes morales (administrations, associations, sociétés…) vous devez, afin de comptabiliser distinctement les opérations réalisées avec lui, lui ouvrir une subdivision d’un compte général appelé aussi compte auxiliaire (ou compte nominatif).

\sphinxAtStartPar
Exemple : vous achetez régulièrement des fournitures de bureau à la SARL « Papeterie du Centre » dont la gérante est Mme VERGER Louise
\begin{description}
\sphinxlineitem{Sous DIACAMMA :}\begin{description}
\sphinxlineitem{Fonction}{[}Gérante                                  \textendash{}\textgreater{}{]}\begin{description}
\sphinxlineitem{Personne physique}{[}VERGER Louise{]}\begin{description}
\sphinxlineitem{Personne morale}{[}Papeterie du Centre (SARL) avec pour membre VERGER Louise, gérante{]}
\sphinxAtStartPar
Tiers comptable : Papeterie du Centre \sphinxhyphen{} Code 401

\end{description}

\end{description}

\end{description}

\end{description}


\subsection{Création d’un tiers comptable}
\label{\detokenize{accounting/third:creation-d-un-tiers-comptable}}\begin{quote}

\sphinxAtStartPar
Menu \sphinxstyleemphasis{Comptabilité/Tiers}
\end{quote}

\noindent{\hspace*{\fill}\sphinxincludegraphics[height=400\sphinxpxdimen]{{third_list}.png}\hspace*{\fill}}

\sphinxAtStartPar
La liste des tiers précédemment enregistrés s’affiche à l’écran. Dans cette liste, notez la présence des membres de votre structure.
Vous pouvez filtrer cette liste par contact ou type de tiers et imprimer la liste obtenue.

\sphinxAtStartPar
Pour ajouter un nouveau tiers comptable, cliquez sur « + Créer » et sélectionnez le contact (personne physique ou morale) associé à ce tiers comptable.

\noindent{\hspace*{\fill}\sphinxincludegraphics[height=400\sphinxpxdimen]{{third_add}.png}\hspace*{\fill}}

\sphinxAtStartPar
Depuis cet écran, vous pouvez aussi créer un nouveau contact à l’aide du bouton « + Créer » avant de le sélectionner.

\noindent\sphinxincludegraphics{{third_edit}.png}

\sphinxAtStartPar
Il est possible d’associer un tiers comptable à un ou plusieurs comptes comptables : fournisseur, client…. Cela permet en comptabilité de distinguer les opérations selon la nature des échanges que vous avez avec lui.


\subsection{Situation d’un tiers}
\label{\detokenize{accounting/third:situation-d-un-tiers}}
\sphinxAtStartPar
L’onglet « Ecritures comptables » de la fiche d’un tiers vous donne une vue sur les écritures comptables le concernant,
avec possibilité de les filtrer.

\noindent{\hspace*{\fill}\sphinxincludegraphics[height=400\sphinxpxdimen]{{third_state}.png}\hspace*{\fill}}


\subsection{Champs personnalisés de tiers}
\label{\detokenize{accounting/third:champs-personnalises-de-tiers}}\begin{quote}

\sphinxAtStartPar
Menu \sphinxstyleemphasis{Administration/Modules (conf.)/Configuration comptable} \sphinxhyphen{} onglet « Champ personnalisé de tiers »
\end{quote}

\sphinxAtStartPar
Grâce à l’ajout de vos propres champs dans la fiche type de tiers comptable, vous allez pouvoir personnaliser cette fiche.
La méthode est similaire à celle utilisée pour personnaliser la fiche des contacts.

\sphinxstepscope


\section{Écritures}
\label{\detokenize{accounting/entity:ecritures}}\label{\detokenize{accounting/entity::doc}}
\sphinxAtStartPar
Avant toute saisie d’écriture, assurez\sphinxhyphen{}vous de l’existence de vos journaux :
\begin{quote}

\sphinxAtStartPar
Menu \sphinxstyleemphasis{Administration/Modules(conf.)/Configuration comptable} onglet « Journaux »
\end{quote}

\sphinxAtStartPar
De même, vous devez contrôler que votre plan comptable est à jour.
\begin{quote}

\sphinxAtStartPar
Menu \sphinxstyleemphasis{Comptabilité/Gestion comptable/Plan comptable}
\end{quote}


\subsection{Saisie d’une écriture}
\label{\detokenize{accounting/entity:saisie-d-une-ecriture}}

\subsubsection{Cas général}
\label{\detokenize{accounting/entity:cas-general}}\begin{quote}

\sphinxAtStartPar
Menu \sphinxstyleemphasis{Comptabilité/Gestion comptable/Écritures comptables}
\end{quote}

\noindent{\hspace*{\fill}\sphinxincludegraphics[height=400\sphinxpxdimen]{{entity_list}.png}\hspace*{\fill}}

\sphinxAtStartPar
Depuis cet écran, vous avez la possibilité de visualiser les écritures précédemment saisies et vous pouvez en ajouter de nouvelles.
A l’écran, vous pouvez aussi consulter les écritures saisies après les avoir filtrées sur l’exercice comptable, un journal (ou tous)  et/ou sur l’état des écritures :
\begin{itemize}
\item {} 
\sphinxAtStartPar
Tout : aucun filtrage n’est appliqué

\item {} 
\sphinxAtStartPar
En cours (brouillard): seulement les écritures provisoires (non encore validées)

\item {} 
\sphinxAtStartPar
Validée : seulement les écritures déjà validées

\item {} 
\sphinxAtStartPar
Lettrée : seulement les écritures lettrées

\item {} 
\sphinxAtStartPar
Non lettrée : seulement les écritures non encore lettrées

\end{itemize}

\sphinxAtStartPar
Vous pouvez également indiquer un code comptable pour filtre les écritures associées.

\sphinxAtStartPar
Pour saisir une nouvelle écriture, cliquez sur le bouton « + Ajouter ».
Sélectionnez le journal et saisissez la date de l’opération ainsi que le libellé de l’écriture (pièce et numéro…). Cliquez sur le bouton « Modifier ».

\sphinxAtStartPar
Pour chaque ligne de l’écriture :
\begin{itemize}
\item {} 
\sphinxAtStartPar
Saisissez les premiers chiffres du numéro du compte devant être mouvementé et sélectionnez\sphinxhyphen{}le dans votre plan comptable. Un compte doit exister dans le plan comptable de l’exercice pour être mouvementable

\item {} 
\sphinxAtStartPar
Si demandé, sélectionnez le tiers et le code analytique de rattachement

\item {} 
\sphinxAtStartPar
Saisissez le montant au débit ou au crédit du compte spécifié

\item {} 
\sphinxAtStartPar
Cliquez sur le bouton « + Ajouter ».

\end{itemize}

\sphinxAtStartPar
Ceci fait, cliquez sur « Ok ». Vous ne pourrez valider la saisie de votre écriture que si elle est équilibrée (débits = crédits).

\sphinxAtStartPar
\sphinxstylestrong{Après clôture de l’exercice comptable, il ne sera plus possible de passer une écriture sur l’exercice clos. Avant cela, prenez soin de passer vos dernières écritures.}

\sphinxAtStartPar
Une écriture étant provisoire, le bouton « Inverser » vous permet d’inverser très facilement votre écriture si besoin.
Quand on débute en comptabilité, on a parfois du mal à savoir si un compte doit être débité ou crédité.
En saisie, lorsque vous débitez un compte fournisseur et créditez un compte de charge, un message vous alerte sur le fait que vous avez probablement saisi l »écriture d’un avoir ».


\subsubsection{Comptabiliser un règlement}
\label{\detokenize{accounting/entity:comptabiliser-un-reglement}}
\sphinxAtStartPar
Un règlement peut être saisi manuellement comme précédemment. Mais bien souvent il est lié à une opération déjà comptabilisée  (ex. achat, appel de fonds) et constatant une dette ou une créance.

\sphinxAtStartPar
Pour simplifier votre saisie, éditez l’écriture constatant la dette ou la créance réglée. Cliquez sur le bouton « Règlement » : l’application vous propose alors une nouvelle écriture partiellement remplie avec le compte du tiers débité (règlement de dette) ou crédité (règlement de créance).
Il ne vous reste plus qu’à préciser sur quel compte financier (caisse, banque…) vous voulez imputer le règlement et à contrôler l’écriture générée après l’avoir complétée.

\sphinxAtStartPar
Une fois l’écriture de règlement validée via cette fonctionnalité, l’écriture d’origine de la dette ou de la créance et celle du règlement sont automatiquement associées. Les lignes concernant le compte de tiers seront lettrées automatiquement.


\subsubsection{Écriture d’à\sphinxhyphen{}nouveaux}
\label{\detokenize{accounting/entity:ecriture-d-a-nouveaux}}
\sphinxAtStartPar
Après clôture d’un exercice comptable, l’écriture d’à\sphinxhyphen{}nouveaux est automatiquement générée lors de la phase d’initialisation de l’exercice suivant. Cette écriture, passée dans le journal « Report à nouveau », est automatiquement validée.
A ce moment, vous pouvez être amené à saisir des opérations spécifiques comme par exemple la ventilation des excédents de l’année précédente.

\sphinxAtStartPar
Par contre, dans ce journal, vous ne pouvez pas enregistrer de charges ou de produits.


\subsection{Lettrage d’écritures}
\label{\detokenize{accounting/entity:lettrage-d-ecritures}}
\sphinxAtStartPar
Comme nous l’avons évoqué dans un précédent chapitre, il est fréquent que des mouvements enregistrés en comptabilité trouvent leur source dans une ou plusieurs opérations liées. Dans ce cas, les lignes d’écritures correspondantes peuvent être lettrées.

\sphinxAtStartPar
Toutefois, il y a des conditions pour que des lignes d’écriture soient lettrables :
\begin{itemize}
\item {} 
\sphinxAtStartPar
Elles doivent concerner le même compte de tiers (fournisseur ou copropriétaire)

\item {} 
\sphinxAtStartPar
Pour être lettrables, le total des débits doit être égal au total des crédits

\item {} 
\sphinxAtStartPar
Soit les lignes sont d’un même exercice non clôturé (il faut donc lettrer les comptes de tiers avant clôture de l’exercice)

\item {} 
\sphinxAtStartPar
Soit les lignes sont sur deux exercices non clôturés qui ce suivent. Le lettrage sera alors temporaire (avec un « \& » en suffix) et corrigé définitivement au moment de la clôture et du report à\sphinxhyphen{}nouveau.

\end{itemize}

\sphinxAtStartPar
Les lignes lettrées conjointement se voient attribuer le même code lettre.

\sphinxAtStartPar
Toute ligne lettrée peut être délettrée.


\subsection{Validation d’écritures}
\label{\detokenize{accounting/entity:validation-d-ecritures}}
\sphinxAtStartPar
Par défaut, une écriture est saisie au brouillard, ce qui permet de la modifier ou de la supprimer tant qu’elle n’est pas validée.
Cette écriture doit être validée pour entériner votre saisie. En principe, cette validation est confiée à la personne en charge de la vérification de la comptabilisation des opérations.

\sphinxAtStartPar
Pour réaliser cette action, sélectionnez les écritures contrôlées et cliquez sur le bouton « Clôturer »: L’application affectera alors un numéro aux écritures validées ainsi qu’une date de validation.

\sphinxAtStartPar
Une fois validée, une écriture devient non modifiable : ce mécanisme assure le caractére intangible et irréversible de votre comptabilité.

\sphinxAtStartPar
Comme l’erreur est humaine, l’écriture validée ne pouvant pas être modifiée ou supprimée, vous devrez procéder comme suit :
\begin{itemize}
\item {} 
\sphinxAtStartPar
1 : Contrepasser l’écriture erronée en créant une écriture inverse pour l’annuler. Le libellé doit spécifier la référence de l’écriture annulée avec la mention « Contrepassation… »

\item {} 
\sphinxAtStartPar
2 : Enregistrer l’écriture correcte

\end{itemize}

\sphinxAtStartPar
\sphinxstylestrong{Avant clôture de l’exercice comptable, toutes les écritures doivent étre validées}.


\subsection{Recherche d’écriture(s)}
\label{\detokenize{accounting/entity:recherche-d-ecriture-s}}
\sphinxAtStartPar
Depuis la liste des écritures, le bouton « Recherche » vous permet de définir les critères de recherche d’écritures comptables.

\noindent{\hspace*{\fill}\sphinxincludegraphics[height=400\sphinxpxdimen]{{entity_search}.png}\hspace*{\fill}}

\sphinxAtStartPar
En cliquant sur « Recherche », l’outil va rechercher dans la base toutes les écritures satisfaisant aux critères saisis.
La ou les écritures extraites pourront être :
\begin{itemize}
\item {} 
\sphinxAtStartPar
Imprimées

\item {} 
\sphinxAtStartPar
Éditées/modifiées

\item {} 
\sphinxAtStartPar
Clôturée, lettrées ou délettrées…

\end{itemize}


\subsection{Import d’écritures}
\label{\detokenize{accounting/entity:import-d-ecritures}}
\sphinxAtStartPar
Depuis la liste des écritures, le bouton « Import » vous permet d’importer des écritures comptables depuis un fichier CSV, ODS ou XLS.

\sphinxAtStartPar
Après avoir sélectionné l’exercice d’import, le journal et les informations de format de votre fichier, vous devez associer les champs des écritures aux colonnes de votre document (la première ligne de votre document doit décrire la nature de chaque colonne).

\noindent{\hspace*{\fill}\sphinxincludegraphics[height=400\sphinxpxdimen]{{entity_import}.png}\hspace*{\fill}}

\sphinxAtStartPar
Vous pouvez alors contrôler vos données avant de les valider.
L’import réalisé, l’outil vous présentera le résultat des écritures réellement importées.

\sphinxAtStartPar
\sphinxstylestrong{Notez que les lignes d’écritures ne seront pas importées si :}
\begin{itemize}
\item {} 
\sphinxAtStartPar
Le code comptable précisé n’existe pas dans le plan comptable de l’exercice

\item {} 
\sphinxAtStartPar
La date n’est pas inclue dans l’exercice comptable actif

\item {} 
\sphinxAtStartPar
Le principe de la partie double n’est pas respecté car pour toute opération, le total des débits doit être égal au total des crédits

\end{itemize}

\sphinxAtStartPar
Bien que cela ne bloque pas l’import, le tiers et le code analytique seront laissés vides si ceux indiqués ne sont pas référencés dans votre dossier comptable. Vous devez donc contrôler l’importation et la modifier si besoin.

\sphinxstepscope


\section{Comptabilité analytique}
\label{\detokenize{accounting/costaccounting:comptabilite-analytique}}\label{\detokenize{accounting/costaccounting::doc}}
\sphinxAtStartPar
Pour réaliser une analyse financière des différentes activités de votre structure et déterminer le résultat de chacune d’elles, vous pouvez mettre en place une comptabilité analytique.
\begin{quote}

\sphinxAtStartPar
Menu \sphinxstyleemphasis{Administration/Modules(conf.)/Configuration comptable}
\end{quote}

\sphinxAtStartPar
Après avoir ouvert l’onglet « Paramètres » \sphinxhyphen{} bouton « Modifier », cochez le paramètre \sphinxstyleemphasis{Comptabilité analytique}. A noter qu’en activant ce paramètre, toutes les charges et les produits devront obligatoirement être repris en comptabilité analytique.

\sphinxAtStartPar
La comptabilité analytique proposée par le logiciel est une version simplifiée. En effet, il n’est pas possible de ventiler une charge ou un produit sur plusieurs codes analytiques.


\subsection{Les codes analytiques}
\label{\detokenize{accounting/costaccounting:les-codes-analytiques}}\begin{quote}

\sphinxAtStartPar
Menu \sphinxstyleemphasis{Comptabilité/Gestion comptable/Comptabilités analytiques}
\end{quote}

\sphinxAtStartPar
Vous accédez à la liste des codes analytiques créés.

\sphinxAtStartPar
Depuis cet écran vous pouvez créer, modifier ou supprimer un code analytique.

\sphinxAtStartPar
Chaque code a un titre, un descriptif et un statut (ouvert ou clôturé).
En mode « liste », figure le résultat comptable (les produits diminués des charges) de chaque code analytique.

\noindent{\hspace*{\fill}\sphinxincludegraphics[height=400\sphinxpxdimen]{{costaccount_list}.png}\hspace*{\fill}}

\sphinxAtStartPar
Par défaut, un filtrage vous permet de ne voir que  les codes analytiques ouverts. Vous pouvez paramétrer le filtre à l’aide des boutons de liste.


\subsection{Imputation analytique d’une charge ou d’un produit}
\label{\detokenize{accounting/costaccounting:imputation-analytique-d-une-charge-ou-d-un-produit}}\begin{quote}

\sphinxAtStartPar
Menu \sphinxstyleemphasis{Comptabilité/Gestion comptable/Ecritures comptables}
\end{quote}

\sphinxAtStartPar
Si vous avez des codes analytiques ouverts, vous pouvez imputer une charge ou un produit sur l’un d’entre eux.

\noindent{\hspace*{\fill}\sphinxincludegraphics[height=200\sphinxpxdimen]{{costaccount_assign}.png}\hspace*{\fill}}

\sphinxAtStartPar
Que l’écriture correspondante soit validée ou non, affichez cette écriture  et éditez\sphinxhyphen{}la.
Sélectionnez le  mouvement relatif à la charge ou au produit à imputer en analytique et bouton « Modifier ».
Renseignez le code analytique de rattachement.

\sphinxAtStartPar
Il est aussi possible de réaliser cette imputation par lot.
Sélectionnez les mouvements à affecter et cliquez sur le bouton \sphinxstyleemphasis{Analytique}. Choisissez alors le nouveau code à utiliser
pour l’ensemble des mouvements (charges ou  produits) sélectionnés.


\subsection{Impressions analytiques}
\label{\detokenize{accounting/costaccounting:impressions-analytiques}}
\sphinxAtStartPar
Depuis la liste des codes analytiques, vous pouvez réaliser un rapport type « Compte de résultat de comptabilité analytique ».
Pour cela, affichez les codes analytiques. Sélectionnez\sphinxhyphen{}les (un ou plusieurs) et cliquez sur le bouton « Rapport ».
Tout comme le Compte de résultat de la Comptabilité générale, ce rapport reprend pour chaque code analytique sélectionné, le montant des charges et des produits ainsi que le budget. Cela permet de comparer, là encore le résultat réel d’une activité au résultat attendu.

\sphinxstepscope


\section{Modèle d’écriture}
\label{\detokenize{accounting/model:modele-d-ecriture}}\label{\detokenize{accounting/model::doc}}

\subsection{Déclaration d’un modèle}
\label{\detokenize{accounting/model:declaration-d-un-modele}}
\sphinxAtStartPar
Certaines opérations sont régulièrement enregistrées. Pour soulager la saisie de celles\sphinxhyphen{}ci, un modèle d’écriture ou masque de saisie peut être créé et enregistré afin d’être utilisé ultérieurement en saisie.
Un modèle d’écriture peut aussi être utilisé pour faciliter la comptabilisation d’opérations complexes.

\sphinxAtStartPar
Le modèle se présente comme une écriture.
\begin{quote}

\sphinxAtStartPar
Menu \sphinxstyleemphasis{Comptabilité/Gestion comptable/Modèles d’écritures}
\end{quote}

\noindent{\hspace*{\fill}\sphinxincludegraphics[height=400\sphinxpxdimen]{{model_list}.png}\hspace*{\fill}}

\sphinxAtStartPar
Il est associé à un journal, contient un descriptif, précise les comptes à débiter et ceux à créditer, avec indication des montants qui ne peuvent pas être laissés nuls.

\noindent{\hspace*{\fill}\sphinxincludegraphics[height=400\sphinxpxdimen]{{model_item}.png}\hspace*{\fill}}

\sphinxAtStartPar
Nous vous conseillons de créer un modèle pour chacune de vos dépenses ou recettes règulières. Ainsi, vous gagnerez du temps sur la saisie de votre comptabilité et n’aurez pas à rechercher les bons codes comptables ni le sens des imputations.


\subsection{Utilisation d’un modèle}
\label{\detokenize{accounting/model:utilisation-d-un-modele}}
\sphinxAtStartPar
L’utilisation d’un modèle est très simple.

\sphinxAtStartPar
En saisie d’écriture, cliquez sur le bouton « + Modèle ».

\noindent{\hspace*{\fill}\sphinxincludegraphics[height=200\sphinxpxdimen]{{model_add}.png}\hspace*{\fill}}

\sphinxAtStartPar
Sélectionnez votre modèle et précisez le coefficient multiplicateur qui devra être appliqué au montant présaisi dans le modèle. Ce facteur est très pratique lorsque l’on a des factures récurrentes mais dont le montant peut fluctuer. Il est alors possible, è l’aide de ce réel, de le faire varier.
Validez votre sélection par « Ok ». Une écriture est générée d’après le modèle. Vous pouvez la corriger comme n’importe quelle écriture.

\sphinxstepscope


\section{Budgets prévisionnels}
\label{\detokenize{accounting/budget:budgets-previsionnels}}\label{\detokenize{accounting/budget::doc}}

\subsection{Budget analytique}
\label{\detokenize{accounting/budget:budget-analytique}}\begin{quote}

\sphinxAtStartPar
Menu \sphinxstyleemphasis{Comptabilité/Gestion comptable/Comptabilités analytiques}
\end{quote}

\sphinxAtStartPar
Depuis l’interface des comptabilités analytiques, vous pouvez ajouter un budget prévisionnel à chacun.

\sphinxAtStartPar
Cliquez simplement sur le bouton « Budget » après avoir sélectionné la comptabilité à compléter.

\sphinxAtStartPar
L’interface vous permet alors d’ajouter (comptes et montants) des charges ou des produits ; le budget peut être en déséquilibre.
Vous pouvez également importer les montants des charges et produits réels d’une comptabilité précédente.

\sphinxAtStartPar
Ce budget prévisionnel apparait alors dans les rapports afin de le comparer avec le réalisé.


\subsection{Budget prévisionnel général}
\label{\detokenize{accounting/budget:budget-previsionnel-general}}\begin{quote}

\sphinxAtStartPar
Menu \sphinxstyleemphasis{Comptabilité/Gestion comptable/Plan comptable}
\end{quote}

\sphinxAtStartPar
Depuis l’interface du plan comptable courant, vous pouvez renseigner le budget prévisionnel de l’exercice via le bouton « Budget ».

\sphinxAtStartPar
Comme pour un budget analytique, vous pouvez ajouter des comptes de charges ou de produits ainsi qu’importer le réalisé d’un exercice précédent.
A noter qu’automatiquement, l’ensemble des budgets analytiques associés au même exercice sont automatiquement consolidés dans le budget prévisionnel général.

\sphinxAtStartPar
Le \sphinxstyleemphasis{Compte de résultat} reprend également le budget prévisionnel à des fins de comparaison.

\sphinxstepscope


\section{Rapports}
\label{\detokenize{accounting/reporting:rapports}}\label{\detokenize{accounting/reporting::doc}}\begin{quote}

\sphinxAtStartPar
Menu \sphinxstyleemphasis{Comptabilité/Rapports comptables}
\end{quote}

\sphinxAtStartPar
Dans cette catégorie, vous accédez à l’ensemble des documents de synthèse élaborés en fin d’exercice. Les rapports sont obtenus à partir de tous les enregistrements comptables passés pendant l’exercice comptable.

\sphinxAtStartPar
En cours d’exercice, tout rapport peut être consulté à l’écran, sauvegardé dans le gestionnaire de documents au format PDF ou imprimé. Ultérieurement, vous pourrez aussi consulter un rapport sauvegardé et en réaliser une impression tel qu’il a été sauvegardé ou en le régénérant sur un autre modèle.

\sphinxAtStartPar
A la clôture de l’exercice, l’ensemble de ces rapports sont générés et sauvegardés automatiquement dans le gestionnaire de documents. Lorsque vous les éditerez, par défaut vous téléchargerez la sauvegarde. Vous pourrez régénérer un nouveau PDF sur un autre modèle. Par contre celui\sphinxhyphen{}ci comportera la mention « duplicata » en filigrane.

\sphinxAtStartPar
Dans le bilan comptable et le compte de résultat, le budget de l’exercice est reporté afin de permettre la comparaison réalisé\sphinxhyphen{}prévisionnel.


\subsection{Compte de résultat}
\label{\detokenize{accounting/reporting:compte-de-resultat}}
\sphinxAtStartPar
Le compte de résultat  synthétise l’ensemble des charges et des produits de l’exercice comptable.
Il met en évidence le résultat net, c’est\sphinxhyphen{}à\sphinxhyphen{}dire la différence entre vos produits et vos charges. Il y a excédent quand les produits excèdent les charges et inversement, un déficit.


\subsection{Bilan comptable}
\label{\detokenize{accounting/reporting:bilan-comptable}}
\sphinxAtStartPar
Le bilan comptable est une photographie (c’est un instantané) du patrimoine de l’entreprise qui permet l’évaluation d’une structure, et plus précisément de savoir après retraitement (par exemple d’une optique patrimoniale à celle sur option de juste valeur pour l’adoption des normes internationales) combien elle vaut et si elle est solvable.
Au bilan, le résultat qui y figure est égal au solde du compte de résultat (excédent ou déficit)

\sphinxAtStartPar
Il existe trois finalités au bilan comptable :
\begin{itemize}
\item {} 
\sphinxAtStartPar
Le bilan interne, généralement détaillé, utilisé par les responsables de la structure pour différentes analyses internes

\item {} 
\sphinxAtStartPar
Le bilan officiel, destiné aux contrôleurs de la comptabilité (auditeurs et commissaires aux comptes) et aux tiers (actionnaires, banques, clients, salariés, collectivités…).

\item {} 
\sphinxAtStartPar
Le bilan fiscal qui sert à déterminer le bénéfice imposable

\end{itemize}


\subsection{Grand livre}
\label{\detokenize{accounting/reporting:grand-livre}}
\sphinxAtStartPar
Le grand livre est le recueil de l’ensemble des comptes utilisés par une structure dans le cadre de la tenue de sa comptabilité.
Il faut distinguer le grand livre général (comptes des classes 1 à 7) des grands livres auxiliaires avec le détail des comptes de tiers (clients, fournisseurs, associés, copropriétaires).

\sphinxAtStartPar
vous pouvez paramétrer vos éditions afin de les personnaliser, avec :
\begin{itemize}
\item {} 
\sphinxAtStartPar
La période (dates de début et de fin)

\item {} 
\sphinxAtStartPar
Le code comptable commençant par

\item {} 
\sphinxAtStartPar
Ecritures non\sphinxhyphen{}lettrées ou toutes

\end{itemize}


\subsection{Balance}
\label{\detokenize{accounting/reporting:balance}}
\sphinxAtStartPar
Elle est établie pour une période donnée, à partir de tous les comptes du grand livre de la struture. Tout comme le grand livre, elle aussi peut être générale (comptes des classes 1 à 7) ou auxilaire (comptes de tiers). Elle regroupe toutes les montants enregistrés au débit et au crédit de ces comptes et par différence tous les soldes débiteurs et créditeurs.

\sphinxAtStartPar
La balance générale doit être équilibrée, avec « total des débits » = « total des crédits » et « total des soldes débiteurs » = « total des soldes créditeurs ». Cet équilibre permet de vérifier que le principe de la partie double a bien été respecté lors de la comptabilisation des opérations et que tous les mouvements passés au journal ont bien été reportés dans le grand libre.

\sphinxAtStartPar
Vous pouvez personnaliser vos éditions avec :
\begin{itemize}
\item {} 
\sphinxAtStartPar
L’exercice comptable

\item {} 
\sphinxAtStartPar
La période (dates de début et de fin)

\item {} 
\sphinxAtStartPar
Le code comptable commençant par

\item {} 
\sphinxAtStartPar
Seulement les (comptes) non soldés

\end{itemize}

\sphinxAtStartPar
Vous pouvez aussi spécifier si le détail par tiers est souhaité, ce qui permet de transformer votre balance générale en balance mixte générale\sphinxhyphen{}auxiliaire.


\subsection{Listing des écritures}
\label{\detokenize{accounting/reporting:listing-des-ecritures}}
\sphinxAtStartPar
Aux rapports comptables, s’ajoute l’édition des journaux :
\begin{quote}

\sphinxAtStartPar
Menu \sphinxstyleemphasis{Comptabilité/Gestion comptable/Ecritures comptables}
\end{quote}

\sphinxAtStartPar
Depuis l’écran de la liste des écritures comptables, vous avez la possibilité de les visualiser et de les exporter en PDF ou au format CSV ou ODS (ce qui permet l’import dans un tableur).


\subsection{Listing du plan comptable de l’exercice}
\label{\detokenize{accounting/reporting:listing-du-plan-comptable-de-l-exercice}}
\sphinxAtStartPar
Depuis l’écran du plan comptable de l’exercice :
\begin{quote}

\sphinxAtStartPar
Menu \sphinxstyleemphasis{Comptabilité/Gestion comptable/Plan comptable}
\end{quote}

\sphinxAtStartPar
Pour un exercice donné et par type de comptes (ou tous), vous pouvez visualiser, pour l’ensemble des comptes ouverts, le récapitulatif des soldes de début et de fin d’exercice avec l’indication des soldes de fin compte\sphinxhyphen{}tenu des seules écritures validées.

\sphinxAtStartPar
Ce récapitulatif peut être imprimé, exporté au format PDF, CSV ou ODS (ce qui permet l’import de vos soldes dans un tableur).

\sphinxstepscope


\chapter{Diacamma facturation}
\label{\detokenize{invoice/index:diacamma-facturation}}\label{\detokenize{invoice/index::doc}}
\sphinxAtStartPar
Aide relative aux fonctionnalités de gestion de factures.

\sphinxstepscope


\section{Les articles}
\label{\detokenize{invoice/articles:les-articles}}\label{\detokenize{invoice/articles::doc}}

\subsection{Création et modification}
\label{\detokenize{invoice/articles:creation-et-modification}}
\sphinxAtStartPar
Depuis le menu \sphinxstyleemphasis{Facturier/Les articles} vous avez la possibilité de définir l’ensemble de vos articles facturables.

\noindent{\hspace*{\fill}\sphinxincludegraphics[height=400\sphinxpxdimen]{{articles_list}.png}\hspace*{\fill}}

\sphinxAtStartPar
Vous pouvez ajouter, modifier ou supprimer un article. La suppression n’est pas possible si l’article est utilisé dans une facture.

\sphinxAtStartPar
A chaque article, vous devez définir un code comptable d’imputation pour la génération d’écritures automatique.
\begin{description}
\sphinxlineitem{Le champ \sphinxstyleemphasis{stockable} permet de définir si vous voulez gérer une gestion de stock de cet article:}\begin{itemize}
\item {} \begin{description}
\sphinxlineitem{non stockable}
\sphinxAtStartPar
Article sans gestion de stock, comme par exemple des articles de service.

\end{description}

\item {} \begin{description}
\sphinxlineitem{stockable}
\sphinxAtStartPar
Article stockable et facturable.

\end{description}

\item {} \begin{description}
\sphinxlineitem{stockable \& non vendable}
\sphinxAtStartPar
Article stockable non proposable à la vente.
Utile pour suivre des stocks de matériel interne.

\end{description}

\item {} \begin{description}
\sphinxlineitem{kit d’article}
\sphinxAtStartPar
Cet article est constitué d’un ensemble d’articles stockables.
Il est nécessaire de définir une liste d’article le consituant ainsi que la quantité dans la recette.
Cet article n’a pas de stock lui\sphinxhyphen{}même, il est déduit des stocks des articles associés.
À la facturation, c’est la quantité des articles associés qui sont alors sortie du stock.

\end{description}

\end{itemize}

\end{description}

\sphinxAtStartPar
De plus, dans le cas où vous réalisez des factures avec TVA, vous devrez préciser, pour chaque articles, le taux de taxe à appliquer.

\sphinxAtStartPar
L’onglet \sphinxstyleemphasis{Fournisseur} permet d’identifier des références fournisseurs pour simplifier leur commande ou leur référencement.

\sphinxAtStartPar
Le champ \sphinxstyleemphasis{code d’imputation comptable} permet d’associer à cet article une configuration comptable (voir \sphinxstyleemphasis{Configuration et paramétrage})


\subsection{La facture avec TVA}
\label{\detokenize{invoice/articles:la-facture-avec-tva}}
\sphinxAtStartPar
Si vous êtes soumis à la TVA, l’édition de la facture change un peu

\sphinxAtStartPar
En plus de préciser si les articles sont en montant HT ou TTC, vous avez en bas de l’écran le total de la facture hors\sphinxhyphen{}taxe, taxes comprises ainsi que le montant total des taxes.

\sphinxAtStartPar
De plus, dans l’impression de la facture, un sous\sphinxhyphen{}détail des taxes apparait par taux de TVA.


\subsection{Import d’articles}
\label{\detokenize{invoice/articles:import-d-articles}}
\sphinxAtStartPar
Depuis le menu \sphinxstyleemphasis{Administration/Modules (conf.)/Import d’article}, vous avez la possibilité d’importer des articles en lot depuis un fichier CSV, ODS ou XLS.

\sphinxAtStartPar
Une fois avoir sélectionné votre fichier et les information de son format,
vous serez ammené à associer les champs d’un article aux colonnes de votre documents (la première ligne de votre document doit décrire la nature de chaque colonne).

\sphinxAtStartPar
Vous pouvez alors contrôler vos données avant de les validés.

\sphinxstepscope


\section{Gestion du facturier}
\label{\detokenize{invoice/create_bill:gestion-du-facturier}}\label{\detokenize{invoice/create_bill::doc}}

\subsection{Type de justificatif}
\label{\detokenize{invoice/create_bill:type-de-justificatif}}
\sphinxAtStartPar
Dans le facturier \sphinxstyleemphasis{Diacamma}, il y a 5 type différents de justificatif de facturation:
\begin{quote}
\begin{itemize}
\item {} 
\sphinxAtStartPar
\sphinxstylestrong{Panier}

\end{itemize}

\sphinxAtStartPar
Le type \sphinxstyleemphasis{panier} n’est activable que depuis le menu \sphinxstyleemphasis{Administration/Modules (conf.)/Configuration commercial du facturier}, onglet \sphinxstyleemphasis{Panier}.
Il permet l’utilisation d’un outil de gestion d’achat en ligne: remplissage d’un panier d’achat.

\sphinxAtStartPar
Une fois activé, un nouveau menu \sphinxstyleemphasis{General/Panier d’achat} apparait alors à tout les utilisateurs ayant le droit spécifique « {[}Facturier Diacamma{]} facture : panier ».
Par défaut, ce type (et l’écran associé) n’est pas actif.

\sphinxAtStartPar
Une fois le document « panier » rempli et validé par l’acheteur, celui\sphinxhyphen{}ci est transformé en \sphinxstyleemphasis{Devis} (un par lieu de stockage différent si les articles sont stockable).
\begin{itemize}
\item {} 
\sphinxAtStartPar
\sphinxstylestrong{Devis}

\end{itemize}

\sphinxAtStartPar
Le \sphinxstyleemphasis{devis} permet d’envoyer à un acheteur potentiel, un document indiquant l’estimation du prix d’un ensemble d’articles de services ou de biens.

\sphinxAtStartPar
Un \sphinxstyleemphasis{devis} peut\sphinxhyphen{}être créé directement depuis le facturier \sphinxstyleemphasis{Diacamma} ou venant d’une transformation d’un \sphinxstyleemphasis{panier}.
Un \sphinxstyleemphasis{devis} peut\sphinxhyphen{}être transformer en \sphinxstyleemphasis{commande} “validée” ou en \sphinxstyleemphasis{facture} “en création”.
\begin{itemize}
\item {} 
\sphinxAtStartPar
\sphinxstylestrong{Commande}

\end{itemize}

\sphinxAtStartPar
Le type \sphinxstyleemphasis{commande} n’est activable que depuis le menu \sphinxstyleemphasis{Administration/Modules (conf.)/Configuration commercial du facturier}, onglet \sphinxstyleemphasis{Paramètres}.

\sphinxAtStartPar
Une \sphinxstyleemphasis{commande} est une démarche de vente résultant d’un \sphinxstyleemphasis{devis} accepté formellement par un acheteur.
C’est une mise en attente de réalisation suite à des contraintes logistique (transport) ou d’approvisionnement avant de pouvoir être transformer en \sphinxstyleemphasis{facture}.

\sphinxAtStartPar
Par contre, contrairement à un \sphinxstyleemphasis{devis}, un règlement peut lui être associée (prouvant l’acceptation de l’acheteur).

\sphinxAtStartPar
Une \sphinxstyleemphasis{commande} ne peux venir que d’un devis, ne peux pas être à l’état “en création” et ne peux qu’être transformer en \sphinxstyleemphasis{facture}.
\begin{itemize}
\item {} 
\sphinxAtStartPar
\sphinxstylestrong{Facture}

\end{itemize}

\sphinxAtStartPar
Une \sphinxstyleemphasis{facture} est le justificatif d’achat prouvant la vente d’un ensemble d’articles.
Un ensemble de paiement peux lui\sphinxhyphen{}etre attachée.

\sphinxAtStartPar
Une \sphinxstyleemphasis{facture} peut\sphinxhyphen{}être créé directement depuis le facturier \sphinxstyleemphasis{Diacamma} ou venant d’une transformation d’un \sphinxstyleemphasis{devis}, d’une \sphinxstyleemphasis{commande} ou d’un \sphinxstyleemphasis{avoir}.
Elle peux également être transformer en \sphinxstyleemphasis{avoir} (droit spécifique « {[}Facturier Diacamma{]} facture : avoir »)
\begin{itemize}
\item {} 
\sphinxAtStartPar
\sphinxstylestrong{Avoir}

\end{itemize}

\sphinxAtStartPar
Un \sphinxstyleemphasis{avoir} est le justificatif prouvant le remboursement d’un ensemble d’articles.
Un ensemble de paiement peux lui\sphinxhyphen{}etre attachée.

\sphinxAtStartPar
Une \sphinxstyleemphasis{avoir} peut\sphinxhyphen{}être créé directement depuis le facturier \sphinxstyleemphasis{Diacamma} ou venant d’une transformation d’une \sphinxstyleemphasis{facture}.
Il peux également être transformer en \sphinxstyleemphasis{facture} (droit spécifique « {[}Facturier Diacamma{]} facture : avoir »)
\end{quote}


\subsection{Création}
\label{\detokenize{invoice/create_bill:creation}}
\sphinxAtStartPar
Depuis le menu \sphinxstyleemphasis{Facturier/Facture} vous pouvez éditer ou ajouter une nouvelle facture.

\sphinxAtStartPar
Commencez par définir le type de document (devis, facture, reçu ou avoir) que vous souhaitez créer ainsi que la date d’émission et un commentaire qui figurera dessus.

\sphinxAtStartPar
Dans cette facture, vous devez préciser le client associé, c’est à dire le tiers comptable imputable de l’opération.

\noindent{\hspace*{\fill}\sphinxincludegraphics[height=400\sphinxpxdimen]{{bill_edit}.png}\hspace*{\fill}}

\sphinxAtStartPar
Ensuite ajoutez ou enlevez autant d’articles que vous le désirez.

\noindent{\hspace*{\fill}\sphinxincludegraphics[height=400\sphinxpxdimen]{{add_article}.png}\hspace*{\fill}}

\sphinxAtStartPar
Par défaut, vous obtenez la désignation et le prix par défaut de l’article sélectionné, mais l’ensemble est modifiable. Vous pouvez choisir aussi l’article divers: aucune information par défaut n’est alors proposé.

\sphinxAtStartPar
Si l’article a été défini comme \sphinxstyleemphasis{stockable}, vous devrez en plus préciser depuis quel lieu de stockage il sera sortie.
Il n’est bien sur pas possible de vendre plus d’article stockable que l’on possède dans le stock.


\subsection{Création en lot}
\label{\detokenize{invoice/create_bill:creation-en-lot}}
\sphinxAtStartPar
Depuis le menu \sphinxstyleemphasis{Facturier/Facture}, vous pouvez créer des factures en lot via le bouton « En lot ».

\sphinxAtStartPar
Pour cela, vous devez définir un « critère sauvegarder » de recherche de tiers (à réaliser via le menu « Comptabilité \textgreater{} Tiers » \sphinxhyphen{} bouton « Recherche »).
Précisez également le type de document (devis, facture, avoir ou reçu) à réaliser ainsi que l’article à inclure (un seul possible).

\sphinxAtStartPar
Valider la création: la génération des justificatifs, liés aux tiers définis dans la recherche, sera alors réalisée.
Ces justificatifs seront à l’état « en création »: vous pouvez donc les controler avant de les valider et les envoyer.


\subsection{Changement d’état}
\label{\detokenize{invoice/create_bill:changement-d-etat}}
\sphinxAtStartPar
Depuis le menu \sphinxstyleemphasis{Facturier/Facture} vous pouvez consulter les factures en cours, validé ou fini.

\sphinxAtStartPar
Un devis, une facture, un reçu ou un avoir dans l’état « en cours » est un document en cours de conception et il n’est pas encore envoyé au client.

\sphinxAtStartPar
Depuis la fiche du document, vous pouvez le valider: il devient alors imprimable et non modifiable.

\sphinxAtStartPar
Dans ces deux cas, une écriture comptable est alors automatiquement générée.

\sphinxAtStartPar
Un devis validé peut facilement être transformé en facture dans le cas de son acceptation par votre client. La facture ainsi créé se retrouve alors dans l’état « en cours » pour vous permettre de la réajuster.

\sphinxAtStartPar
Une fois qu’une facture (ou un avoir) est considéré comme terminée (c’est à dire réglée ou définie comme pertes et profits), vous pouvez définir son état à «fini».

\sphinxAtStartPar
Depuis une facture « fini », il vous est possible de créer un avoir correspondant à l’état « en cours ». Cette fonctionnalité vous sera utile si vous êtes amené à rembourser un client d’un bien ou un service précédemment facturé.

\sphinxAtStartPar
Vous pouvez également archiver des justificatifs (devis, facture, avoir) quand ceux\sphinxhyphen{}ci ont été gérés (droit spécifique « {[}Facturier Diacamma{]} facture : archivé »).
Notez que si un avoir et un facture sont totalement lié, c’est à dire qu’ils n’ont qu’un règlement interne de l’un vers l’autre, leurs états sont forcés à l’identique entre « validé » et « archivé ».

\sphinxAtStartPar
Si une facture contiens des articles \sphinxstyleemphasis{stockable}, un bordereau de sortie est automatiquement générer pour correspondre à cette vente.
La situation du stock est alors mise à jour automatiquement.


\subsection{Impression}
\label{\detokenize{invoice/create_bill:impression}}
\sphinxAtStartPar
Depuis la fiche d’un document (devis, facture, reçu ou avoir) vous pouvez à tout moment imprimer ou réimprimer celui\sphinxhyphen{}ci s’il n’est pas à l’état «en cours».

\sphinxAtStartPar
Pour les factures, les reçus ou les avoirs, une sauvegarde officielle (d’après le patron d’impression par défaut) est automatiquement sauvegardée dans le gestionnaire de documents au moment de la validation.
Lorsque vous voulez imprimer le justificatif, on vous propose alors par défaut de télécharger ce document sauvegardé.
Vous pouvez régénérer un nouveau PDF, par exemple avec un autre patron d’impression. Par contre celui\sphinxhyphen{}ci comportera la mention « duplicata » en filigrane.


\subsection{Paiement}
\label{\detokenize{invoice/create_bill:paiement}}
\sphinxAtStartPar
Si ceux\sphinxhyphen{}ci sont configurés (menu \sphinxstyleemphasis{Administration/Modules (conf.)/Configuration du règlement}), vous pouvez consulter les moyens de paiement d’une facture, d’un reçu ou d’un devis.
Si vous l’envoyez par courriel, vous pouvez également les faire apparaitre dans votre message.

\sphinxAtStartPar
Dans le cas d’un paiement via PayPal, MoneticoPaiement ou Hello\sphinxhyphen{}Asso, si votre \sphinxstyleemphasis{Diacamma} est accessible par internet, le logiciel sera automatiquement notifié du règlement.
Dans le cas d’un devis, celui\sphinxhyphen{}ci sera automatiquement archivé et une facture équivalente sera générée.
Un nouveau réglement sera ajouté dans votre facture.

\sphinxAtStartPar
Dans l’écran \sphinxstyleemphasis{Comptabilité/Transactions bancaires}, vous pouvez consulté précisement la notification reçu de PayPal, MoneticoPaiement ou Hello\sphinxhyphen{}Asso.
En cas d’état « échec », la raison est alors précisé: il vous faudra manuellement vérifier votre compte (PayPal, MoneticoPaiement ou Hello\sphinxhyphen{}Asso) et rétablir l’éventuellement paiment erroné manuellement.

\sphinxstepscope


\section{Gestion de stock}
\label{\detokenize{invoice/stock:gestion-de-stock}}\label{\detokenize{invoice/stock::doc}}

\subsection{Bordereau de stockage}
\label{\detokenize{invoice/stock:bordereau-de-stockage}}
\sphinxAtStartPar
Depuis le menu \sphinxstyleemphasis{Facturier/Stockage/Bordereau de stockage} vous pouvez créer des bordereaux de reception ou sortie d’article stockable.

\sphinxAtStartPar
Pour cela, vous devez préciser un lieu de stockage qui sera impacté par le mouvement de stock (configurable depuis \sphinxstyleemphasis{Administration/Modules (conf.)/Configuration du facturier})
Dans le cas d’une réception, vous pouvez optionnellement préciser une référence de fournisseur (raison social, facture)
Ajoutez alors les articles ainsi que la valeur du mouvement.
Pour un borderaux de sortie, il n’est pas possible de saisir un montant suppérieur au stock courrant.


\subsection{Situation}
\label{\detokenize{invoice/stock:situation}}
\sphinxAtStartPar
Depuis le menu \sphinxstyleemphasis{Facturier/Stockage/Situation} vous pouvez consulter, pour chaque lieu de stockage, la quantité de chaque article géré.

\sphinxAtStartPar
Le bouton \sphinxstyleemphasis{imprimer} permet de sortir un rapport au format PDF, CSV ou ODS (importable via un tableur)

\sphinxAtStartPar
Depuis la fiche d’un article, vous pouvez également consulter sa situation de stockage pour chacun des lieux de stockage défini.


\subsection{Historique}
\label{\detokenize{invoice/stock:historique}}
\sphinxAtStartPar
Depuis le menu \sphinxstyleemphasis{Facturier/Stockage/Situation} vous pouvez consulter, le mouvement des articles en réception et en sortie, pour une période de temps données.

\sphinxAtStartPar
Le bouton \sphinxstyleemphasis{imprimer} permet également le même type de rapport que précédement.


\subsection{Inventaire}
\label{\detokenize{invoice/stock:inventaire}}
\sphinxAtStartPar
Depuis le menu \sphinxstyleemphasis{Facturier/Stockage/Bordereau d’inventaire} vous pouvez réaliser un inventaire de votre stock.

\sphinxAtStartPar
Pour réaliser un inventaire, il faut commencer par créer un bordereau associé au lieu de stockage que l’on veut inventorier.
De là, le bordereau est alors automatiquement remplis avec l’ensemble des articles stockables référencés.

\sphinxAtStartPar
Vous ne pouvez valider votre bordereau d’inventaire qu’une fois la quantité de chaque articles resaisi.
\sphinxstyleemphasis{astuce} : Le bouton \sphinxstyleemphasis{Copie}, utilisable sur plusieurs ligne sélectionné, permet d’automatiquement assigné la quantité actuelle.

\sphinxAtStartPar
Une fois votre bordereau correctement remplis, et une fois validé, 2 bordereaux de strockages peuvent être alors automatiquement généré afin de réaliser le réajustement (en plus ou en moins) de vos stocks.

\sphinxAtStartPar
Le bouton \sphinxstyleemphasis{imprimer} permet d’éditer en PDF, en CSV ou en ODS la list de l’inventaire.

\sphinxstepscope


\section{Statistiques des ventes}
\label{\detokenize{invoice/statistics:statistiques-des-ventes}}\label{\detokenize{invoice/statistics::doc}}
\noindent{\hspace*{\fill}\sphinxincludegraphics[height=400\sphinxpxdimen]{{statistiques}.png}\hspace*{\fill}}

\sphinxAtStartPar
Vous avez la possibilité d’éditer les statistiques des ventes d’un exercice donné.

\sphinxAtStartPar
Ce document vous donne les sommes et le pourcentage par clients, par articles et par mois.
Il est possible également d’avoir la répartition des réglements par mode.

\sphinxAtStartPar
Via le bouton \sphinxstyleemphasis{Contrôler}, il vous est possible d’avoir un contôle de cohérence entre vos montants vendu et vos lignes de comptabilités.
Cela vous permet de vérifier si une correction spécifique aurait desynchronisé financièrement les deux outils.

\sphinxstepscope


\section{Configuration et paramétrage}
\label{\detokenize{invoice/configuration:configuration-et-parametrage}}\label{\detokenize{invoice/configuration::doc}}

\subsection{Catégories}
\label{\detokenize{invoice/configuration:categories}}
\sphinxAtStartPar
Le menu \sphinxstyleemphasis{Administration/Modules (conf.)/Configuration commercial du facturier}, onglet \sphinxstyleemphasis{Catégorie}

\sphinxAtStartPar
Vous pouvez définir des catégories afin de classer vos articles.
Chaque article peut être associé à plusieurs catégories.


\subsection{Champ personnalisé}
\label{\detokenize{invoice/configuration:champ-personnalise}}
\sphinxAtStartPar
Le menu \sphinxstyleemphasis{Administration/Modules (conf.)/Configuration commercial du facturier}, onglet \sphinxstyleemphasis{Champ personnalisé}

\sphinxAtStartPar
Comme pour les contacts, vous pouvez ici définir des champs personnalisés.


\subsection{Lieu de stockage}
\label{\detokenize{invoice/configuration:lieu-de-stockage}}
\sphinxAtStartPar
Le menu \sphinxstyleemphasis{Administration/Modules (conf.)/Configuration commercial du facturier}, onglet \sphinxstyleemphasis{Lieu de stockage}

\sphinxAtStartPar
Si vous voulez gérer une centrale d’achat, vous pouvez ici définir les différents espace de vos articles stockables.


\subsection{Catégorie de facturation}
\label{\detokenize{invoice/configuration:categorie-de-facturation}}
\sphinxAtStartPar
Le menu \sphinxstyleemphasis{Administration/Modules (conf.)/Configuration commercial du facturier}, onglet \sphinxstyleemphasis{Catégorie de facturation}

\sphinxAtStartPar
Vous pouvez définir des catégories de facturation afin de modifier le nom du type de vos justificatif ou personnaliser vos patrons d’impression.
Il vous est également possible de préciser que les devis/factures de cette catégorie auront une numérotation propre (avec un éventuelle préfixe de spécification.
Les moyens de paiement proposés ainsi que le message par défaut lors de l’envoie par courriel peuvent également être personnalisé.

\sphinxAtStartPar
Notez que si vous définissez des catégories de facturation, la liste des facture (menu \sphinxstyleemphasis{Facturier/Facture}) vous proposera un filtrage par type spécifique de chaque catégorie.


\subsection{Gestion de panier}
\label{\detokenize{invoice/configuration:gestion-de-panier}}
\sphinxAtStartPar
Le menu \sphinxstyleemphasis{Administration/Modules (conf.)/Configuration commercial du facturier}, onglet \sphinxstyleemphasis{Panier}

\sphinxAtStartPar
Vous pouvez ici activer la fonctionnalité d’achat par panier.
Un nouveau menu \sphinxstyleemphasis{General/Panier d’achat} apparait alors à tout les utilisateurs ayant le droit spécifique « {[}Facturier Diacamma{]} facture : panier ».
\begin{description}
\sphinxlineitem{D’autres paramètres permet d’affiner l’a fonctionnalité}\begin{itemize}
\item {} 
\sphinxAtStartPar
Filtre d’article pour le panier : défini un critère de recherche sauvegardé des articles vendables par panier.

\item {} 
\sphinxAtStartPar
Temps de vie d’un panier (jours : défini le temps de conversation (en jours) d’un panier non validé. laissez « 0 » si vous ne voulez pas les nettoyer automatiquement.

\item {} 
\sphinxAtStartPar
sujet et message pour courriel de panier : permet de personnaliser le courriel envoyé à l’acheteur une fois le panier validé.

\end{itemize}

\end{description}


\subsection{Multi\sphinxhyphen{}prix}
\label{\detokenize{invoice/configuration:multi-prix}}
\sphinxAtStartPar
Le menu \sphinxstyleemphasis{Administration/Modules (conf.)/Configuration commercial du facturier}, onglet \sphinxstyleemphasis{Multi\sphinxhyphen{}prix}

\sphinxAtStartPar
Vous pouvez définir, ici, des descriptions de prix multiple.
Ajouter un nouveau sous\sphinxhyphen{}prix en précisant un nom et un critère de filtrage du tiers à qui ce prix s’applique.

\sphinxAtStartPar
On retrouve alors, dans chaque fiche article, l’indication de ce nouveau prix.
Par défaut, celui\sphinxhyphen{}ci est égal au prix générique.

\sphinxAtStartPar
Ce nouveau prix est alors proposé par défaut, dans une nouvelle facture, si le tiers de la facture correspond au critère de celui\sphinxhyphen{}ci.
Si plusieurs sous\sphinxhyphen{}prix sont valides, c’est celui qui propose un prix le plus bas qui est défini.

\sphinxAtStartPar
\sphinxstyleemphasis{Remarque:} si au moins un sous\sphinxhyphen{}prix est configuré, il deviens obligatoir de définir le tiers d’une facture avant d’y ajouter un article.


\subsection{Réduction automatique}
\label{\detokenize{invoice/configuration:reduction-automatique}}
\sphinxAtStartPar
Le menu \sphinxstyleemphasis{Administration/Modules (conf.)/Configuration commercial du facturier}, onglet \sphinxstyleemphasis{Réduction automatique}
\begin{description}
\sphinxlineitem{Ce tableau de gestion de réductions comporte les champs suivants:}\begin{itemize}
\item {} 
\sphinxAtStartPar
La catégorie d’article impacté.

\item {} 
\sphinxAtStartPar
Le type de réduction: en valeur, en pourcentage, en pourcentage global déjà vendu.

\item {} 
\sphinxAtStartPar
Le montant de la réduction (en valeur ou pourcentage suivant le type).

\item {} 
\sphinxAtStartPar
Le nombre d’occurrence déclenchant la réduction.

\item {} 
\sphinxAtStartPar
Un critère de filtrage du tiers à qui cette réduction s’applique. (Optionnel)

\end{itemize}

\end{description}

\sphinxAtStartPar
Au moment d’ajout d’un article dans une facture, si le client de cette facture et ce nouvel article répond à aux critères d’une réduction,
celle\sphinxhyphen{}ci s’applique alors automatiquement dans la facture.
Si plusieurs réductions remplissent leurs conditions, c’est la réduction octroyant la plus grande réduction qui sera utilisé.
Un article peut se retrouver vendu gratuitement, mais jamais négativement (qui reviendrait à un remboursement)
Ce mécanisme sera également appliqué lors de la création automatique des factures (cotisation, participation à un événement)
Ce mécanisme vérifie le critère que pour des opérations réalisés sur l’exercice financier courant de l’association (les réductions ne se cumule pas d’une année à l’autre)


\subsection{Codes d’imputations comptable}
\label{\detokenize{invoice/configuration:codes-d-imputations-comptable}}
\sphinxAtStartPar
Menu \sphinxstyleemphasis{Administration/Modules (conf.)/Configuration financière du facturier}, onglet \sphinxstyleemphasis{Codes d’imputations comptable}
\begin{description}
\sphinxlineitem{Un « Code d’imputation comptable » contiens:}\begin{itemize}
\item {} 
\sphinxAtStartPar
un code comptable de vente

\item {} 
\sphinxAtStartPar
un code analytique (optionnel)

\end{itemize}

\end{description}

\sphinxAtStartPar
Chaque article peut être associé à un code d’imputation comptable (si non précisé, l’article n’est pas vendable).
Ce mécanisme permet de centraliser à un seul endroit les configurations comptables des articles.
Au changement d’exercice, si ces configurations doivent changées, il est plus simple de modifier cette configuration que l’ensemble des articles.

\sphinxAtStartPar
Un champs spécifique peut également être demandé si la gestion des \sphinxstyleemphasis{comande} est autorisée: « compte de tiers de provision ».
Ce compte de tiers sera utilisé par la commande (plutôt que le code comptable habituel « client ») dans le cas où l’on sait que les futures commandes seront transformé en facture dans un laps de temps long (ex: approvisionnement d’article).


\subsection{Codes comptables par défaut}
\label{\detokenize{invoice/configuration:codes-comptables-par-defaut}}
\sphinxAtStartPar
Menu \sphinxstyleemphasis{Administration/Modules (conf.)/Configuration financière du facturier}, onglet \sphinxstyleemphasis{Paramètres}

\sphinxAtStartPar
Ce module est intimement lié au module de gestion comptable, un certain nombre de codes comptables par défaut sont nécessaires.

\sphinxAtStartPar
Pour pouvoir générer les écritures comptables correspondants aux factures saisies avec des articles non référencés, vous devez préciser le code comptable de vente (classe 7) lié a ce type de d’article. Par défaut, le code comptable de vente de service est défini.

\sphinxAtStartPar
Pour réaliser une réduction sur un article, vous devez préciser le code comptable de vente à imputer de cette réduction.
Dans le cas de règlement en liquide, il vous faut préciser le code comptable de banque associé à votre caisse.


\subsection{La configuration de la TVA}
\label{\detokenize{invoice/configuration:la-configuration-de-la-tva}}
\sphinxAtStartPar
Menu \sphinxstyleemphasis{Administration/Modules (conf.)/Configuration financière du facturier}, onglet \sphinxstyleemphasis{TVA}

\sphinxAtStartPar
Vous pouvez complètement configurer la gestion de votre soumission à la TVA.

\noindent{\hspace*{\fill}\sphinxincludegraphics[height=400\sphinxpxdimen]{{vat}.png}\hspace*{\fill}}

\sphinxAtStartPar
Pour commencer, vous devez définir les modalités de soumission en sélectionnant votre mode d’application:
\begin{itemize}
\item {} \begin{description}
\sphinxlineitem{TVA non applicable}
\sphinxAtStartPar
Vous n’êtes pas soumis à la TVA. L’ensemble de vos factures sont réalisées hors\sphinxhyphen{}taxe.

\end{description}

\item {} \begin{description}
\sphinxlineitem{Prix HT}
\sphinxAtStartPar
Vous êtes soumis à la TVA. Vous faites le choix d’éditer vos factures avec les montants des articles en hors\sphinxhyphen{}taxe.

\end{description}

\item {} \begin{description}
\sphinxlineitem{Prix TTC}
\sphinxAtStartPar
Vous êtes soumis à la TVA. Vous faites le choix d’éditer vos factures avec les montants des articles toutes taxes comprises.

\end{description}

\end{itemize}

\sphinxAtStartPar
Précisez également le compte comptable d’imputation de ces taxes.

\sphinxAtStartPar
Définissez l’ensemble des taux de taxes auxquels vos ventes sont soumises. La taxe par défaut sera celle appliquée à l’article libre (sans référence).

\sphinxstepscope


\chapter{Diacamma règlement}
\label{\detokenize{payoff/index:diacamma-reglement}}\label{\detokenize{payoff/index::doc}}
\sphinxAtStartPar
Aide relative aux fonctionnalités de gestion des payements.

\sphinxstepscope


\section{Règlement}
\label{\detokenize{payoff/payoff:reglement}}\label{\detokenize{payoff/payoff::doc}}
\sphinxAtStartPar
Depuis un module tel que la facturation (\sphinxstyleemphasis{Menu Facturier/Facture}) ou la gestion de copropriétaire (\sphinxstyleemphasis{Menu Copropriété/Les propriétaires et les lots}, onglet \sphinxstyleemphasis{Appels et règlements}), il vous est possible de gérer des règlement.

\sphinxAtStartPar
Depuis la fiche du document, cliquez sur «ajouter» un règlement.

\noindent{\hspace*{\fill}\sphinxincludegraphics[height=400\sphinxpxdimen]{{payoff}.png}\hspace*{\fill}}

\sphinxAtStartPar
Dans la boîte de dialogue, saisissez le détail du règlement reçu du tiers en précisant le mode de paiement.
7 modes de paiement vous sont proposés:
\begin{itemize}
\item {} 
\sphinxAtStartPar
espèces

\item {} 
\sphinxAtStartPar
chèque

\item {} 
\sphinxAtStartPar
virement

\item {} 
\sphinxAtStartPar
prélèvement

\item {} 
\sphinxAtStartPar
carte de crédit

\item {} 
\sphinxAtStartPar
autre

\item {} 
\sphinxAtStartPar
interne

\end{itemize}

\sphinxAtStartPar
Le mode \sphinxstyleemphasis{espèces} sera lié à un mouvement comptable avec le code référencé dans \sphinxstyleemphasis{Administration/Configuration des règlements} (onglet \sphinxstyleemphasis{Paramètres}, champ « compte de caisse »).
Pour les modes \sphinxstyleemphasis{chèque}, \sphinxstyleemphasis{virement}, \sphinxstyleemphasis{prélèvement}, \sphinxstyleemphasis{carte de crédit} et \sphinxstyleemphasis{autre}, vous devez préciser le compte bancaire sur lequel imputer ce mouvement financier.
Le mode \sphinxstyleemphasis{interne} permet d’utiliser un autre justificatif financier (comme par exemple un avoir) pour « solder » celui\sphinxhyphen{}ci. Un rapporchement ou une écriture de compensation sera alors réalisé en comptabilité.

\sphinxAtStartPar
Vous pourrez voir alors dans les documents liés le net\sphinxhyphen{}à\sphinxhyphen{}payer et le ou les règlements correspondants ainsi que le reste dû.

\sphinxAtStartPar
Chaque règlement génère automatiquement une écriture comptable au journal.

\sphinxAtStartPar
Il est également possible de réaliser un règlement multiple.
Suivant le type de document sur lequel ce paiement est associé, vous pouvez avoir plusieurs modes de répartition:
\begin{itemize}
\item {} 
\sphinxAtStartPar
Par date : le paiement est d’abord ventilé sur la facture la plus ancienne, puis la suivante, etc.

\item {} 
\sphinxAtStartPar
Par prorata : le paiement est automatiquement ventilé sur toutes les factures sélectionnées, au prorata de leurs montants.

\end{itemize}

\noindent{\hspace*{\fill}\sphinxincludegraphics[height=400\sphinxpxdimen]{{multi-payoff}.png}\hspace*{\fill}}

\sphinxAtStartPar
Quelque soit le mode de répartition, une seule écriture comptable d’encaissement sera alors générées.

\sphinxstepscope


\section{Dépôts de chèques}
\label{\detokenize{payoff/deposit:depots-de-cheques}}\label{\detokenize{payoff/deposit::doc}}\begin{quote}

\sphinxAtStartPar
Menu \sphinxstyleemphasis{Comptabilité/Dépôt de chèques}
\end{quote}

\sphinxAtStartPar
Grâce à ce menu, vous pouvez préparer vos remises de chèques à l’encaissement.
A l’écran s’affiche la liste des bordereaux de dépôt préparés que vous pouvez filtrer sur le statut.

\noindent{\hspace*{\fill}\sphinxincludegraphics[height=400\sphinxpxdimen]{{depositlist}.png}\hspace*{\fill}}

\sphinxAtStartPar
Pour faciliter la recherche d’écritures, vous pouvez les filtrer.


\subsection{Création}
\label{\detokenize{payoff/deposit:creation}}
\sphinxAtStartPar
Si vous souhaitez préparer une nouvelle remise, cliquez sur « + Créer ».
Dans la boîte de dialogue ouverte, sélectionnez le code bancaire concerné, la référence du dépôt et la date de la remise.

\sphinxAtStartPar
Le détail du dépôt de chèques est complété avec le bouton « + Ajouter ».
Sélectionnez ensuite dans la liste des créances, celles réglées et cliquez sur le bouton « Sélectionner » afin que les chèques correspondants soient repris dans le détail du bordereau.

\noindent{\hspace*{\fill}\sphinxincludegraphics[height=400\sphinxpxdimen]{{deposititem}.png}\hspace*{\fill}}

\sphinxAtStartPar
Votre détail étant maintenant actualisé, vous devez « Clôturer » votre dépôt de chèques après l’avoir contrôlé.


\subsection{Impression}
\label{\detokenize{payoff/deposit:impression}}
\sphinxAtStartPar
Avant de clôturer votre bordereau, vous avez la possibilité de faire une première impression.
Par contre, comme votre bordereau n’est pas finalisé, un filigramm « NON VALIDÉ » apparaitra sur le document.

\sphinxAtStartPar
Une fois cloturé, vous pouvrez imprimer votre dépôt de chèques de façon officiel.
Vous pouvez procéder à la remise effective en banque des chèques assortis de ce document de dépôt.


\subsection{Validation}
\label{\detokenize{payoff/deposit:validation}}
\sphinxAtStartPar
En cliquant sur le bouton « validé » vous changer l’état de votre bordereau indiquant que celui\sphinxhyphen{}ci à été réellement crédité sur votre compte bancaire.
Si le \sphinxstyleemphasis{code bancaire} utilisé comporte un ensemble de 2 comptes (bancaire + attente) une écriture comptable sera alors générée afin de mettre à jour votre compte bancaire.

\sphinxstepscope


\section{Configuration}
\label{\detokenize{payoff/config:configuration}}\label{\detokenize{payoff/config::doc}}\begin{quote}

\sphinxAtStartPar
Menu \sphinxstyleemphasis{Administration/Modules (conf.)/Configuration du règlement}
\end{quote}

\sphinxAtStartPar
Ce menu vous permet de configurer les comptes courants et les moyens de paiement en usage dans votre structure.


\subsection{Code bancaire}
\label{\detokenize{payoff/config:code-bancaire}}
\sphinxAtStartPar
Dans cet écran, vous avez la possibilité d’enregistrer les références des différents comptes courants ouverts au nom de votre structure.
A l’aide du bouton « + Ajouter », vous pouvez en créer un nouveau et spécifier : la désignation, le RIB (référence), le code bancaire associé, le journal de mouvement désiré et optionnellement un compte d’attente.
Les compte doivent être déjà ouvert dans le plan comptable de l’exercice comme compte de trésorerie.

\sphinxAtStartPar
Dans le cas où vous préciseriez un compte d’attente, les mouvemets comptables générés seront un peu différent.
Cette option est à utiliser dans les règlements par chèque bancaire (ou assimilé comme des « bons » ou « tickets » subventionnés proposés par certaines collectivités locales).

\sphinxAtStartPar
En effet, au moment de la saisi de votre règlement, l’écriture sera alors créé sur le compte d’attente.
Vous serez alors invité d’utiliser l’outil de gestion de bordereaux de chèques et lors de la validation finale de celui\sphinxhyphen{}ci,
une écriture générale réalisera le mouvement financier entre ce compte d’attente et le compte bancaire proprement dit.


\subsection{Moyen de paiement}
\label{\detokenize{payoff/config:moyen-de-paiement}}
\sphinxAtStartPar
Vous pouvez ici préciser les moyens de paiement que vous acceptez.

\sphinxAtStartPar
Actuellement, 6 moyens de paiement sont possibles sous \sphinxstyleemphasis{Diacamma}, pour chacun d’entre eux, vous devez préciser les paramètres correspondants:
\begin{quote}
\begin{itemize}
\item {} 
\sphinxAtStartPar
Le virement bancaire

\item {} 
\sphinxAtStartPar
Le chèque

\item {} 
\sphinxAtStartPar
Le paiement PayPal

\end{itemize}

\sphinxAtStartPar
Dans le cas de PayPal, si votre \sphinxstyleemphasis{Diacamma} est accessible via le Net, le paiement de votre débiteur vous est notifié sous votre application et le règlement correspondant généré directement dans votre logiciel.
Il est conseillé de cocher le champ \sphinxstyleemphasis{avec contrôle}. Ainsi, le lien de paiement présenté dans un courriel s’assurera dans un premier temps que, sous votre \sphinxstyleemphasis{Diacamma}, l’élément financier est toujours d’actualité.
\begin{itemize}
\item {} 
\sphinxAtStartPar
Le règlement en ligne, via une adresse internet

\item {} 
\sphinxAtStartPar
Le paiement MoneticoPaiement

\end{itemize}

\sphinxAtStartPar
Pour MoneticoPaiement, la plateforme de paiement du Crédit Mutuel et du CIC (voir \sphinxurl{https://www.monetico-paiement.fr}), veuillez prendre contact avec votre agence bancaire pour faire ouvrir votre accès.
Et si vous \sphinxstyleemphasis{Diacamma} est accessible via le Net, vous devez déclarer à la plateforme une adresse de retour de type « \sphinxurl{https://mon.site.diacamma/diacamma.payoff/validationPaymentMoneticoPaiement} » ( »\sphinxurl{https://mon.site.diacamma} » étant l’adresse d’accès à votre serveur \sphinxstyleemphasis{Diacamma})
\begin{itemize}
\item {} 
\sphinxAtStartPar
Le paiement Hello\sphinxhyphen{}Asso

\end{itemize}

\sphinxAtStartPar
Pour utiliser la plateforme Hello\sphinxhyphen{}Asso pour réaliser vos paiements, connectez vous sur le portail \sphinxurl{https://www.helloasso.com}.
Sur votre compte de structure, dans le menu « Mon compte \textgreater{} Intégration et API », vous devez recopiez les champs \sphinxstyleemphasis{mon clientid} et \sphinxstyleemphasis{mon clientsecret} dans l’interface.
Et si votre \sphinxstyleemphasis{Diacamma} est accessible via le Net, le paiement de votre débiteur vous est notifié sous votre application et le règlement correspondant généré directement dans votre logiciel.
\end{quote}

\sphinxAtStartPar
Ces moyens de paiement peuvent être utilisés par vos débiteurs, via précision en bas de couriel, pour régler ce qu’ils vous doivent.


\subsection{Paramètres}
\label{\detokenize{payoff/config:parametres}}
\sphinxAtStartPar
4 Paramètres actuellement :
\begin{itemize}
\item {} 
\sphinxAtStartPar
compte de caisse : code comptable à mouvementer pour les règlements en espèces

\item {} 
\sphinxAtStartPar
compte de frais bancaires : code comptable devant être utilisé pour comptabiliser les frais bancaires liés aux règlements. Une ligne d’écriture est alors ajoutée directement à l’écriture comptable correspondante si le champ « compte de frais bancaires » est renseigné, ce qui permet de saisir le montant des frais.

\item {} 
\sphinxAtStartPar
sujet et message par défaut proposés à l’envoie des justificatifs d’un paiement.

\end{itemize}

\sphinxstepscope


\chapter{Lucterios contacts}
\label{\detokenize{contacts/index:lucterios-contacts}}\label{\detokenize{contacts/index::doc}}
\sphinxAtStartPar
Aide relative aux fonctionnalités de gestion de contacts moraux ou physiques.

\sphinxstepscope


\section{Les contact physiques (personnes physiques)}
\label{\detokenize{contacts/individual:les-contact-physiques-personnes-physiques}}\label{\detokenize{contacts/individual::doc}}
\sphinxAtStartPar
Un contact physique est une personne, homme ou femme, avec qui votre structure est en relation. Il peut s’agir d’un adhérent, d’un copropriétaire mais aussi d’un contact, personne physique, de l’un de vos interlocuteurs (exemple le commercial d’un de vos fournisseurs).
\begin{quote}

\sphinxAtStartPar
Menu \sphinxstyleemphasis{Bureautique/Adresses et Contacts/Personnes physiques}
\end{quote}


\subsection{Liste de vos contacts physiques}
\label{\detokenize{contacts/individual:liste-de-vos-contacts-physiques}}
\sphinxAtStartPar
La liste des personnes déjà enregistrées s’affiche. Étant donné que celle\sphinxhyphen{}ci peut devenir importante, il est possible de filtrer les personnes sur le nom.

\sphinxAtStartPar
Depuis cet écran, vous avez aussi la possibilité d’imprimer la liste des personnes et les étiquettes pour le courrier.

\noindent{\hspace*{\fill}\sphinxincludegraphics[height=400\sphinxpxdimen]{{ListIndividual}.png}\hspace*{\fill}}


\subsection{Visualisation d’un contact physique}
\label{\detokenize{contacts/individual:visualisation-d-un-contact-physique}}
\sphinxAtStartPar
La liste des personnes physiques étant affichée, le bouton « Editer » ou un double\sphinxhyphen{}clic sur la ligne correspondante au contact, permettent de visualiser la fiche du contact.

\noindent{\hspace*{\fill}\sphinxincludegraphics[height=400\sphinxpxdimen]{{ShowIndividual}.png}\hspace*{\fill}}

\sphinxAtStartPar
La fiche consultée peut être modifiée, bouton « modifier » et imprimée, bouton « Imprimer ».
Si cette personne n’est pas associée à d’autres enregistrements de l’application, vous avez la possibilité de supprimer sa fiche.

\sphinxAtStartPar
Vous pouvez également attribuer au contact physique un alias de connexion à l’application, assorti de droits et de permissions (voir Les utilisateurs).

\noindent{\hspace*{\fill}\sphinxincludegraphics[height=400\sphinxpxdimen]{{PermissionsIndividual}.png}\hspace*{\fill}}


\subsection{Ajout d’un contact physique}
\label{\detokenize{contacts/individual:ajout-d-un-contact-physique}}
\sphinxAtStartPar
Depuis la liste précédente, vous avez aussi la possibilité d’ajouter une nouvelle personne à l’aide du bouton « + Créer ».

\noindent{\hspace*{\fill}\sphinxincludegraphics[height=400\sphinxpxdimen]{{EditIndividual}.png}\hspace*{\fill}}


\subsection{Recherche d’un contact physique}
\label{\detokenize{contacts/individual:recherche-d-un-contact-physique}}\begin{quote}

\sphinxAtStartPar
Menu Bureautique/Adresses et Contacts/Recherche de personne physique
\end{quote}

\sphinxAtStartPar
Définissez les critères de recherche grâce à quoi seront extraites de la base toutes les fiches y satisfaisant.
Vous pourrez alors imprimer cette liste ou visualiser/modifier une fiche.

\noindent{\hspace*{\fill}\sphinxincludegraphics[height=400\sphinxpxdimen]{{FindIndividual}.png}\hspace*{\fill}}

\sphinxAtStartPar
Les critères de filtre peuvent être sauvegardés pour une utilisation ultérieure.

\sphinxstepscope


\section{Les contacts moraux (personnes morales)}
\label{\detokenize{contacts/legal_entity:les-contacts-moraux-personnes-morales}}\label{\detokenize{contacts/legal_entity::doc}}
\sphinxAtStartPar
Entreprises, associations, établissements publics, …, les personnes morales sont des groupements de personnes dotées de la personnalité juridique. Elles peuvent être composées d’une ou de plusieurs personnes, individus ou personnes morales.
Pour chacune des personnes morales avec qui votre structure est en contact, une fiche peut être tenue sous \sphinxstyleemphasis{Diacamma}.
\begin{quote}

\sphinxAtStartPar
Menu \sphinxstyleemphasis{Bureautique/Adresses et Contacts/Personnes morales}
\end{quote}


\subsection{Liste de vos contacts moraux}
\label{\detokenize{contacts/legal_entity:liste-de-vos-contacts-moraux}}
\sphinxAtStartPar
Vous pouvez consulter la liste des structures pour lesquelles une fiche a déjà été enregistrée.
Chaque contact moral est associé à une catégorie grâce à quoi vous pouvez filtrer la liste des contacts sur le type de structures.

\noindent{\hspace*{\fill}\sphinxincludegraphics[height=400\sphinxpxdimen]{{ListLegalEntity}.png}\hspace*{\fill}}

\sphinxAtStartPar
Depuis cet écran, vous avez aussi la possibilité d’imprimer la liste des structures avec le bouton « Liste » et pouvez imprimer les étiquettes pour le courrier.

\sphinxAtStartPar
Une nouvelle fiche est ouverte à l’aide du bouton « + Créer » et toute fiche peut être supprimée, à la condition de ne pas être associée à un autre enregistrement de votre base (exemple une facture saisie).


\subsection{Visualisation d’un contact moral}
\label{\detokenize{contacts/legal_entity:visualisation-d-un-contact-moral}}
\sphinxAtStartPar
Depuis la liste précédente, la fiche d’une structure peut être visualisée à l’aide du bouton « Editer » ou d’un double\sphinxhyphen{}clic sur la ligne correspondante au contact.

\noindent{\hspace*{\fill}\sphinxincludegraphics[height=400\sphinxpxdimen]{{ShowLegalEntity}.png}\hspace*{\fill}}

\sphinxAtStartPar
Cette fiche peut ensuite être imprimée avec le bouton du même nom.


\subsection{Modifier un contact moral}
\label{\detokenize{contacts/legal_entity:modifier-un-contact-moral}}
\sphinxAtStartPar
La fiche étant toujours à l’écran, utilisez le bouton « Modifier » pour y apporter toute correction.

\noindent{\hspace*{\fill}\sphinxincludegraphics[height=400\sphinxpxdimen]{{EditLegalEntity}.png}\hspace*{\fill}}


\subsection{Responsables d’un contact moral}
\label{\detokenize{contacts/legal_entity:responsables-d-un-contact-moral}}
\sphinxAtStartPar
Vous avez la possibilité d’associer une personne physique à un contact moral : onglet « Membres » et bouton « + Ajouter ».
Sélectionnez la personne physique. Si elle n’est pas répertorié dans votre base, vous avez la possibilité d’y pourvoir.
Tout nouveau membre peut être assorti d’une fonction.

\noindent{\hspace*{\fill}\sphinxincludegraphics[height=400\sphinxpxdimen]{{ResponsabilityLegalEntity}.png}\hspace*{\fill}}


\subsection{Recherche d’un contact moral}
\label{\detokenize{contacts/legal_entity:recherche-d-un-contact-moral}}
\sphinxAtStartPar
Le menu \sphinxstyleemphasis{Bureautique/Adresses et Contacts/Recherche de personne morale} vous permet d’extraire de votre base les personnes morales satisfaisant aux critères saisis. Ces critères peuvent être sauvegardés pour une utilisation ultérieure.

\noindent{\hspace*{\fill}\sphinxincludegraphics[height=400\sphinxpxdimen]{{FindLegalEntity}.png}\hspace*{\fill}}

\sphinxstepscope


\section{Les possessions}
\label{\detokenize{contacts/possession:les-possessions}}\label{\detokenize{contacts/possession::doc}}
\sphinxAtStartPar
Une possession, est une entité qui appartiens à une personne (moral ou physique).
Cette possession peut être de nature différente:
\begin{quote}
\begin{itemize}
\item {} 
\sphinxAtStartPar
Un animal, comme par exemple un chien, un chat ou un cheval, …

\end{itemize}

\sphinxAtStartPar
Il peut donc être interessant pour une structure lié au monde animal de vouloir cartographié des animaux avec leur maitre et maitresse.
\begin{itemize}
\item {} 
\sphinxAtStartPar
Un véhicule, comme par exemple une voiture, une moto, un kart, ….

\end{itemize}

\sphinxAtStartPar
Ces possessions sont alors interessantes à être identifier afin de connaitre correctement les différents propriétaires.
\end{quote}

\sphinxAtStartPar
Sur ces possessions, il est possible de faire des recherches et en particulier lié aux propriétaires.


\subsection{Configuration}
\label{\detokenize{contacts/possession:configuration}}
\sphinxAtStartPar
Dans le menu \sphinxstyleemphasis{Administration/Configuration des possessions} vous pouvez configurer cette fonctionnalité.

\sphinxAtStartPar
Ajoutez des catégories afin de répartir vos possessions.
Notez que l’absence de catégorie désactive la fonctionnalité.

\sphinxAtStartPar
Ajoutez également des champs personnalisés.
Notez que par défaut, une posséssion ne comporte qu’un nom, une description et une photo.


\subsection{Utilisation}
\label{\detokenize{contacts/possession:utilisation}}
\sphinxAtStartPar
Une fois la configuration réalisée, 2 nouveaux menus apparaient : « Bureautique \textgreater{} Possession » et « Bureautique \textgreater{} Recherche de possessions ».

\sphinxAtStartPar
Dans le premier, vous avez la possibilité de voir la liste de possessions ainsi que d’ajouter, modifier et supprimer une possession.
Notons également qu’il est possible de gérer les possessions directement depuis la fiche d’un contact.

\sphinxAtStartPar
L’outil de recherche permet donc de définir un crière de recherche. Le critère peux également inclure des caractèristiques d’un contact.
De plus, depuis la recherche de contacts, il est également possible d’ajouter un critère lié au possession.

\sphinxstepscope


\section{Configuration et paramétrage}
\label{\detokenize{contacts/configuration:configuration-et-parametrage}}\label{\detokenize{contacts/configuration::doc}}
\sphinxAtStartPar
Dans le menu \sphinxstyleemphasis{Administration/Modules (conf.)} vous avez à votre disposition des outils pour configurer la gestion des contacts.


\subsection{Configuration des contacts}
\label{\detokenize{contacts/configuration:configuration-des-contacts}}
\sphinxAtStartPar
Dans cet écran, vous avez la possibilité de créer ou de modifier la liste des fonctions et des responsabilités que peuvent avoir les contacts  associés aux personnes physiques et morales avec lesquelles vous êtes en relation. Vous pouvez aussi créer ou modifier les types de structures pour vous aider à classer vos personnes morales.

\sphinxAtStartPar
Il se peut que vous ayez besoin de préciser des informations supplémentaires pour vos différents contacts. Vous avez ici la possibilité d’ajouter des champs personnalisés pour chaque type de contacts. Pour ajouter un champ, vous devez simplement préciser le modèle de contact auquel il s’applique, indiquer son titre ainsi que définir son type et préciser si le champ est multiligne ou non.

\sphinxAtStartPar
Six types sont possibles :
\begin{itemize}
\item {} 
\sphinxAtStartPar
chaîne de texte

\item {} 
\sphinxAtStartPar
nombre entier

\item {} 
\sphinxAtStartPar
nombre à virgule (réel)

\item {} 
\sphinxAtStartPar
valeur Oui/Non (booléen)

\item {} 
\sphinxAtStartPar
sélection dans une liste (énumération)

\item {} 
\sphinxAtStartPar
date

\end{itemize}

\sphinxAtStartPar
Dans le cas de l’énumération, vous devez définir la liste des valeurs possibles (mots) séparées par un point\sphinxhyphen{}virgule.


\subsection{Codes postaux/villes}
\label{\detokenize{contacts/configuration:codes-postaux-villes}}
\sphinxAtStartPar
En saisie, l’outil va automatiquement rechercher la ville (ou les villes) associée(s) au code postal que vous entrerez afin de vous faciliter la saisie de vos contacts.
Dans cet écran, vous pouvez ajouter les codes postaux manquants.
Par défaut, les codes postaux français et suisses sont insérés.


\subsection{Import de contacts}
\label{\detokenize{contacts/configuration:import-de-contacts}}
\sphinxAtStartPar
Depuis le menu \sphinxstyleemphasis{Administration/Modules (conf.)/Importation des contacts}, vous avez la possibilité d’importer des contacts en lot depuis un fichier CSV, ODS ou XLS.

\sphinxAtStartPar
Une fois avoir sélectionné votre fichier, le type de contact et les information de son format,
vous serez ammené à associer les champs d’un article aux colonnes de votre documents (la première ligne de votre document doit décrire la nature de chaque colonne).

\sphinxAtStartPar
Vous pouvez alors contrôler vos données avant de les validés.

\sphinxstepscope


\chapter{Lucterios messagerie}
\label{\detokenize{mailing/index:lucterios-messagerie}}\label{\detokenize{mailing/index::doc}}
\sphinxAtStartPar
Aide relative aux fonctionnalités de courier, de publipostage et de SMS.

\sphinxstepscope


\section{Publipostage}
\label{\detokenize{mailing/mailing:publipostage}}\label{\detokenize{mailing/mailing::doc}}\begin{quote}

\sphinxAtStartPar
Menu \sphinxstyleemphasis{Bureautique/Publipostage/Messages courriel}
\end{quote}


\subsection{Création d’un message}
\label{\detokenize{mailing/mailing:creation-d-un-message}}
\sphinxAtStartPar
Bouton « + Ajouter »

\sphinxAtStartPar
Une fois votre message rédigé, validez\sphinxhyphen{}le en cliquant sur « Ok ». En ouvrant l’onglet « Destinataires », vous allez pouvoir saisir le ou les critères permettant de filtrer vos contacts. Le résultat de la requête est affichée à l’écran. Vous pouvez l’affiner et sauvegarder vos critères pour une utilisation ultérieure.

\sphinxAtStartPar
Votre requête prête, cliquez sur « validée ». D’autres requêtes peuvent être ajoutées à la première, leurs résultats se cumulant.
Au moment où votre courrier sera généré, vos requêtes seront de nouveau exécutées, grâce à quoi votre base peut être mise à jour avant \sphinxstyleemphasis{Validation et transmission} des courriers si vous constatez qu’un contact est absent du résultat des requêtes.

\sphinxAtStartPar
Il est également possible de joindre à votre message un ou plusieurs documents sauvegardés dans le \sphinxstyleemphasis{Gestionnaire de documents}. Pour cela, ouvrez l’onglet « Documents ».

\sphinxAtStartPar
L’option \sphinxstyleemphasis{document(s) ajouté(s) via liens dans le message} permet d’ajouter un ensemble de liens de partage vers vos documents (et non plus comme pièces jointes). Cela permet la transmission de documents de taille importante ou qui risqueraient d’être supprimés par certains gestionnaires de courriel.


\subsection{Validation \& transmission}
\label{\detokenize{mailing/mailing:validation-transmission}}\begin{description}
\sphinxlineitem{Une fois le message validé vous pouvez :}\begin{itemize}
\item {} 
\sphinxAtStartPar
Soit générer une sortie PDF de l’ensemble des lettres à envoyer, personnalisées avec l’en\sphinxhyphen{}tête de chaque contact

\item {} 
\sphinxAtStartPar
Soit envoyer votre message par courriel, si celui\sphinxhyphen{}ci est correctement configuré. Bien sur, dans ce cas, seuls les contacts possédant une adresse électronique seront impactés par cet envoi.

\end{itemize}

\end{description}

\sphinxAtStartPar
De plus, dans le cas d’un envoi par courriel, vous pouvez consulter le rapport de transmission. Celui\sphinxhyphen{}ci vous indique les courriels envoyés et les éventuelles erreurs d’acheminement.

\sphinxAtStartPar
Si votre logiciel est accessible depuis internet, vous pouvez également consulter le nombre de fois que le destinataire a consulté ce message.
Ce mécanisme se base sur l’acceptation, par votre destinataire des images distantes présentent dans le message.

\sphinxstepscope


\section{Envoi de SMS}
\label{\detokenize{mailing/sms:envoi-de-sms}}\label{\detokenize{mailing/sms::doc}}\begin{quote}

\sphinxAtStartPar
Menu \sphinxstyleemphasis{Bureautique/Publipostage/Messages SMS}
\end{quote}


\subsection{Création d’un message}
\label{\detokenize{mailing/sms:creation-d-un-message}}
\sphinxAtStartPar
L’envoi de SMS se fait via une interface proche de celle d’envoie de couriels.

\sphinxAtStartPar
Notons les différences:
\begin{itemize}
\item {} 
\sphinxAtStartPar
Spécificité de création de message
\begin{quote}

\sphinxAtStartPar
Vous devez précisez quel champ (tel1, tel2 ou un champ textuel personalisé) vous souhaitez utiliser pour envoyer vos messages.
Notez qu’il est possible d’envoyer à plusieurs numéro pour un même contact.
\end{quote}

\item {} 
\sphinxAtStartPar
La taille du message de celui\sphinxhyphen{}ci s’affiche.
\begin{quote}

\sphinxAtStartPar
Notez que certain caractère, comme les accents, peuvent compter double.
\end{quote}

\item {} 
\sphinxAtStartPar
Il n’est pas possible d’ajouter de document joint comme pour les courriels.

\end{itemize}


\subsection{Validation \& transmission}
\label{\detokenize{mailing/sms:validation-transmission}}
\sphinxAtStartPar
Une fois le message validé vous pouvez l’envoyer par SMS, si celui\sphinxhyphen{}ci est correctement configuré.
Bien sur, dans ce cas, seuls les contacts possédant une numéro de téléphone valide (voir configuration) seront impactés par cet envoi.

\sphinxAtStartPar
De plus, vous pouvez consulter le rapport de transmission.
Celui\sphinxhyphen{}ci vous indique les SMS envoyés et les éventuelles erreurs d’acheminement.

\sphinxstepscope


\section{Configuration de la messagerie}
\label{\detokenize{mailing/configuration:configuration-de-la-messagerie}}\label{\detokenize{mailing/configuration::doc}}\begin{quote}

\sphinxAtStartPar
Menu Administration/Modules (conf.)/Paramètres de courrier \& SMS
\end{quote}

\sphinxAtStartPar
Vous pouvez configurer ici des réglages pour l’envoi de couriel et de SMS.


\subsection{Configuration du couriel}
\label{\detokenize{mailing/configuration:configuration-du-couriel}}
\sphinxAtStartPar
Le serveur SMTP permettra au logiciel d’envoyer directement des messages à vos contacts.
Configurez donc ici les règlages de votre serveur.
Vous pouvez également préciser un \sphinxstyleemphasis{Fichier privé DKIM} et \sphinxstyleemphasis{Sélecteur DKIM} afin de signer vos envois de courriel.
Les paramètres \sphinxstyleemphasis{durée (en min) d’un lot de courriel} et \sphinxstyleemphasis{nombre de courriels par lot} sont utilisés pour l’envoie des messages en publipostage.

\sphinxAtStartPar
Un bouton \sphinxstyleemphasis{Envoyer} permet de tester vos règlages en envoyant un courriel de test à un destinataire choisi.
Il existe des outils permettant de vérifier si vos messages respectent des règles afin d’éviter d’être considéré comme des “pourriel”.
En autre, l’outil \sphinxurl{https://www.mail-tester.com} (gratuit jusqu’à 3 fois par jour) vous permet, en envoyant un message à l’adresse précisée, de vous établir une note de confiance.

\sphinxAtStartPar
Vous pouvez, entre autre, envoyer d’un nouveau mot de passe de connexion.
N’oubliez pas alors de préciser un petit message d’explication via le paramètre \sphinxstyleemphasis{Message de confirmation de connexion}.


\subsection{Configuration du SMS}
\label{\detokenize{mailing/configuration:configuration-du-sms}}
\sphinxAtStartPar
En configurant un fournisseur de SMS, vous pourrez alors envoyer des SMS à vos contacts.

\sphinxAtStartPar
Pour chaque fournisseur, vous devrez préciser le champ « Expression d’analyse de numéro (SMS) ».
Il correspond à une expression régulière afin de savoir comment transformer un numéro de téléphone en numéro international.
Par défaut, il correspond au numéro français: \sphinxstyleemphasis{\textasciicircum{}0({[}67{]}{[}0\sphinxhyphen{}9{]}\{8\})\$|+33\{0\}}
\begin{quote}

\sphinxAtStartPar
Il vérifie que le numéro comporte 10 chiffres.
Il commance par “06” ou “07”.
Il remplacera le “0” par “+33”.
\end{quote}

\sphinxAtStartPar
Suivant les besoins et les demandes, d’autres fournisseurs pourront être ajoutés à l’avenir.


\subsubsection{Mailjet SMS}
\label{\detokenize{mailing/configuration:mailjet-sms}}
\sphinxAtStartPar
Mailjet (\sphinxurl{https://www.mailjet.com/}) est une entreprise française qui propose, entre autres, d’envoyer des SMS.

\sphinxAtStartPar
Pour utiliser ce fournisseur, vous devez créer un compte sur leur site web.

\sphinxAtStartPar
Rendez vous ensuite sur leur portail SMS (\sphinxurl{https://app.mailjet.com/sms}) où vous pourrez configurer votre accès au SMS:
\begin{itemize}
\item {} 
\sphinxAtStartPar
Générer votre token d’accès

\item {} 
\sphinxAtStartPar
Créditer votre solde de SMS prépayé

\end{itemize}

\sphinxAtStartPar
Notez que le coùt d’un SMS dépend de sa taille et de sa destination.

\sphinxAtStartPar
Vous devrez ensuite indiquer dans « Options pour fournisseur SMS » votre « api token » généré précédement
ainsi qu’un « alias » qui sera l’identifiant présent dans le SMS envoyé.

\sphinxAtStartPar
Une fois votre configuration terminer, un bouton \sphinxstyleemphasis{Envoyer} permet de tester vos règlages en envoyant un SMS de contrôle à un numéro choisi.

\sphinxstepscope


\chapter{Lucterios documents}
\label{\detokenize{documents/index:lucterios-documents}}\label{\detokenize{documents/index::doc}}
\sphinxAtStartPar
Aide relative aux fonctionnalités de gestion documentaire.

\sphinxstepscope


\section{Gestion de fichiers}
\label{\detokenize{documents/shared_document:gestion-de-fichiers}}\label{\detokenize{documents/shared_document::doc}}

\subsection{Liste des documents}
\label{\detokenize{documents/shared_document:liste-des-documents}}
\sphinxAtStartPar
Pour retrouver plus aisément vos documents, sous \sphinxstyleemphasis{Diacamma}, ceux\sphinxhyphen{}ci peuvent être enregistrés dans des dossiers et des sous\sphinxhyphen{}dossiers du gestionnaire de documents.
Chaque dossier est assorti d’une description et d’informations relatives à la dernière modification.
\begin{quote}

\sphinxAtStartPar
Menu \sphinxstyleemphasis{Bureautique/Gestion de fichiers et de documents/Documents}
\end{quote}

\sphinxAtStartPar
Le retour au dossier\sphinxhyphen{}parent est possible grâce au bouton “Retour”.
Si vous disposez des droits, le bouton d’édition situé à coté permet d’éditer les propriétés du dossier actif et de les modifier.

\noindent{\hspace*{\fill}\sphinxincludegraphics[height=400\sphinxpxdimen]{{listdoc}.png}\hspace*{\fill}}

\sphinxAtStartPar
Suivant vos permissions, vous pouvez de gérer vos dossiers (modifier ses propriétés ou en ajouter un de plus).
La suppression d’un dossier ne se fait qu’exclusivement depuis l’écran d’administration (cf chapitre « Configurer les dossiers »)
Affichez le contenu d’un dossier après en faisant un double\sphinxhyphen{}clic sur l’élement correspondante au dossier.

\sphinxAtStartPar
Depuis la zone centrale, vous pouvez visualiser les documents contenus dans le dossier courant.
Suivant la nature des fichiers, une miniature peux s’afficher pour vous présenter un apperçu de celui\sphinxhyphen{}ci.
Suivant vos permissions, vous pouvez ajouter ou supprimer un document.
\sphinxstyleemphasis{Diacamma} mémorise l’utilisateur et la date de création de tout document ainsi que les informations relatives à la dernière modification.

\sphinxAtStartPar
Affichez la fiche d’un document à l’aide d’un double\sphinxhyphen{}clic sur son nom.

\noindent{\hspace*{\fill}\sphinxincludegraphics[height=400\sphinxpxdimen]{{showdoc}.png}\hspace*{\fill}}
\begin{description}
\sphinxlineitem{Depuis cette fiche, il vous est possible :}\begin{itemize}
\item {} 
\sphinxAtStartPar
de le modifier en l’important de nouveau

\item {} 
\sphinxAtStartPar
de générer un lien de téléchargement. Ce lien web peut être transmis à une personne n’ayant aucun droit d’accès à votre logiciel afin qu’elle puisse télécharger le document

\end{itemize}

\end{description}

\sphinxAtStartPar
\sphinxstylestrong{Attention:} Votre instance doit être accessible sur internet pour que ce lien puisse fonctionner.


\subsection{Recherche de documents}
\label{\detokenize{documents/shared_document:recherche-de-documents}}\begin{quote}

\sphinxAtStartPar
Menu \sphinxstyleemphasis{Bureautique/Gestion documentaire/Recherche de document}
\end{quote}

\sphinxAtStartPar
Saisissez les critères de recherche de documents et validez. Ces critères sont sauvegardables pour une utilisation ultérieure.

\sphinxAtStartPar
Diacamma parcourera tous les dossiers du gestionnaire de documents afin d’en extraire la liste de ceux satisfaisant aux critères saisis.

\sphinxstepscope


\section{Editeur de documents}
\label{\detokenize{documents/editor:editeur-de-documents}}\label{\detokenize{documents/editor::doc}}
\sphinxAtStartPar
Il est possible de configurer l’outil afin de pouvoir éditer certains documents directement via l’interface « en ligne ».

\sphinxAtStartPar
Des outils d’édition, libres et gratuits, sont actuellement configurables afin de les utiliser pour consulter et modifier des documents.

\sphinxAtStartPar
\sphinxstylestrong{Note :} Ces outils sont gérés par des équipes complètement différentes, il se peut que certains de leurs comportements ne correspondent pas à vos attentes.


\subsection{Etherpad}
\label{\detokenize{documents/editor:etherpad}}
\sphinxAtStartPar
Éditeur pour document textuel.

\sphinxAtStartPar
Site Web
\begin{quote}

\sphinxAtStartPar
\sphinxurl{https://etherpad.org/}
\end{quote}

\sphinxAtStartPar
Installation
\begin{quote}

\sphinxAtStartPar
Le tutoriel de framasoft explique bien comment l’installer: \sphinxurl{https://framacloud.org/fr/cultiver-son-jardin/etherpad.html}
\end{quote}

\sphinxAtStartPar
Configurer
\begin{quote}
\begin{quote}

\sphinxAtStartPar
Éditer le fichier « settings.py » contenu dans le répertoire de votre instance.
Ajouter et adapter la ligne ci\sphinxhyphen{}dessous:
\end{quote}
\begin{itemize}
\item {} 
\sphinxAtStartPar
url : adresse d’accès d’Etherpad

\item {} 
\sphinxAtStartPar
apikey : contenu de la clef de sécurité (fichier APIKEY.txt contenu dans l’installation d’etherpad)

\end{itemize}
\end{quote}

\begin{sphinxVerbatim}[commandchars=\\\{\}]
\PYG{c+c1}{\PYGZsh{} extra}
\PYG{n}{ETHERPAD} \PYG{o}{=} \PYG{p}{\PYGZob{}}\PYG{l+s+s1}{\PYGZsq{}}\PYG{l+s+s1}{url}\PYG{l+s+s1}{\PYGZsq{}}\PYG{p}{:} \PYG{l+s+s1}{\PYGZsq{}}\PYG{l+s+s1}{http://localhost:9001}\PYG{l+s+s1}{\PYGZsq{}}\PYG{p}{,} \PYG{l+s+s1}{\PYGZsq{}}\PYG{l+s+s1}{apikey}\PYG{l+s+s1}{\PYGZsq{}}\PYG{p}{:} \PYG{l+s+s1}{\PYGZsq{}}\PYG{l+s+s1}{jfks5dsdS65lfGHsdSDQ4fsdDG4lklsdq6Gfs4Gsdfos8fs}\PYG{l+s+s1}{\PYGZsq{}}\PYG{p}{\PYGZcb{}}
\end{sphinxVerbatim}

\sphinxAtStartPar
Usage
\begin{quote}
\begin{description}
\sphinxlineitem{Dans le gestionnaire de documents, vous avez plusieurs actions qui apparaissent alors:}\begin{itemize}
\item {} 
\sphinxAtStartPar
Un bouton « + Fichier » vous permettant de créer un document txt ou html depuis la liste d’un dossier.

\item {} 
\sphinxAtStartPar
Un bouton « Editeur » pour ouvrir l’éditeur Etherpad depuis la fiche du document.

\end{itemize}

\end{description}
\end{quote}

\noindent{\hspace*{\fill}\sphinxincludegraphics[height=400\sphinxpxdimen]{{etherpad}.png}\hspace*{\fill}}


\subsection{Ethercalc}
\label{\detokenize{documents/editor:ethercalc}}
\sphinxAtStartPar
Éditeur pour tableau de calcul.

\sphinxAtStartPar
Site Web
\begin{quote}

\sphinxAtStartPar
\sphinxurl{https://ethercalc.net/}
\sphinxurl{https://github.com/audreyt/ethercalc}
\end{quote}

\sphinxAtStartPar
Installation
\begin{quote}

\sphinxAtStartPar
Sur le wiki des sources, une petit explication indique comment l’installer.
\end{quote}

\sphinxAtStartPar
Configurer
\begin{quote}
\begin{quote}

\sphinxAtStartPar
Editer le fichier « settings.py » contenu dans le répertoire de votre instance.
Ajouter et adapter la ligne ci\sphinxhyphen{}dessous:
\end{quote}
\begin{itemize}
\item {} 
\sphinxAtStartPar
url : adresse d’accès d’Ethercal

\end{itemize}
\end{quote}

\begin{sphinxVerbatim}[commandchars=\\\{\}]
\PYG{c+c1}{\PYGZsh{} extra}
\PYG{n}{ETHERCALC} \PYG{o}{=} \PYG{p}{\PYGZob{}}\PYG{l+s+s1}{\PYGZsq{}}\PYG{l+s+s1}{url}\PYG{l+s+s1}{\PYGZsq{}}\PYG{p}{:} \PYG{l+s+s1}{\PYGZsq{}}\PYG{l+s+s1}{http://localhost:8000}\PYG{l+s+s1}{\PYGZsq{}}\PYG{p}{\PYGZcb{}}
\end{sphinxVerbatim}

\sphinxAtStartPar
Usage
\begin{quote}
\begin{description}
\sphinxlineitem{Dans le gestionnaire de documents, vous avez plusieurs actions qui apparaissent alors:}\begin{itemize}
\item {} 
\sphinxAtStartPar
Un bouton « + Fichier » vous permettant de créer un document csv, ods ou xmlx depuis la liste d’un dossier.

\item {} 
\sphinxAtStartPar
Un bouton « Editeur » pour ouvrir l’éditeur Ethercalc depuis la fiche du document.

\end{itemize}

\end{description}
\end{quote}

\noindent{\hspace*{\fill}\sphinxincludegraphics[height=400\sphinxpxdimen]{{ethercalc}.png}\hspace*{\fill}}


\subsection{OnlyOffice Docs Community Edition}
\label{\detokenize{documents/editor:onlyoffice-docs-community-edition}}
\sphinxAtStartPar
Éditeurs en ligne pour les documents texte, les feuilles de calcul et les présentations.

\sphinxAtStartPar
Site Web
\begin{quote}

\sphinxAtStartPar
\sphinxurl{https://www.onlyoffice.com/fr/office-suite.aspx}
\end{quote}

\sphinxAtStartPar
Installation
\begin{quote}

\sphinxAtStartPar
Nous vous recommandons l’installation via Docker comme expliqué ici:
\sphinxurl{https://helpcenter.onlyoffice.com/installation/docs-community-install-docker.aspx}
\end{quote}

\sphinxAtStartPar
Configurer
\begin{quote}

\sphinxAtStartPar
Editer le fichier « settings.py » contenu dans le répertoire de votre instance.
Ajouter et adapter la ligne ci\sphinxhyphen{}dessous:
\begin{itemize}
\item {} 
\sphinxAtStartPar
url : adresse d’accès d’OnlyOffice

\end{itemize}
\end{quote}

\begin{sphinxVerbatim}[commandchars=\\\{\}]
\PYG{c+c1}{\PYGZsh{} extra}
\PYG{n}{ONLYOFFICE} \PYG{o}{=} \PYG{p}{\PYGZob{}}\PYG{l+s+s1}{\PYGZsq{}}\PYG{l+s+s1}{url}\PYG{l+s+s1}{\PYGZsq{}}\PYG{p}{:} \PYG{l+s+s1}{\PYGZsq{}}\PYG{l+s+s1}{http://localhost:8100}\PYG{l+s+s1}{\PYGZsq{}}\PYG{p}{\PYGZcb{}}
\end{sphinxVerbatim}

\sphinxAtStartPar
Usage
\begin{quote}
\begin{description}
\sphinxlineitem{Dans le gestionnaire de documents, vous avez plusieurs actions qui apparaissent alors:}\begin{itemize}
\item {} 
\sphinxAtStartPar
Un bouton « + Fichier » vous permettant de créer un document csv, xlsx, ods, docx, odt, txt, pptx ou odp depuis la liste d’un dossier.

\item {} 
\sphinxAtStartPar
Un bouton « Editeur » pour ouvrir l’éditeur OnlyOffice depuis la fiche du document.

\item {} 
\sphinxAtStartPar
A noter que vous avez également la possibilité de visualiser, en lecture seule, les documents xls, doc, ppt ou pdf

\end{itemize}

\end{description}
\end{quote}

\noindent{\hspace*{\fill}\sphinxincludegraphics[height=400\sphinxpxdimen]{{onlyoffice}.png}\hspace*{\fill}}

\sphinxstepscope


\section{Configurer les dossiers}
\label{\detokenize{documents/configuration:configurer-les-dossiers}}\label{\detokenize{documents/configuration::doc}}
\sphinxAtStartPar
Pour votre gestion documentaire, vous disposez d’un ensemble d’outils.
\begin{quote}

\sphinxAtStartPar
Menu \sphinxstyleemphasis{Administration/Module (conf)/Dossiers}
\end{quote}

\sphinxAtStartPar
A l’écran la liste des dossiers existants s’affiche. Vous avez la possibilité d’en créer de nouveaux et de modifier les paramètres des dossiers déjà présents.

\noindent{\hspace*{\fill}\sphinxincludegraphics[height=400\sphinxpxdimen]{{configuration}.png}\hspace*{\fill}}

\sphinxAtStartPar
En définissant judicieusement un dossier comme sous\sphinxhyphen{}dossier d’un dossier\sphinxhyphen{}parent, vous pouvez  mettre en place une arborescence de dossiers respectueuse de votre plan de classement.

\sphinxAtStartPar
Pour chaque dossier, vous pouvez aussi définir les droits des groupes d’utilisateurs, que cela soit pour la visualisation ou la modification des fichiers qui sont/seront enregistrés dans le dossier.
Ainsi, les utilisateurs appartenant aux groupes de visualisation pourront seulement consulter les documents contenus dans ce dossier. Les utilisateurs appartenant aux groupes de modification pourront, eux, les modifier ou les supprimer.

\sphinxAtStartPar
Le bouton « Extraire » permet de réaliser la sauvegarde d’un dossier et de son contenu sous forme d’une archive au format zip.
Quand au bouton « Importer », il permet d’importer une archive au format zip qui sera décompressée automatiquement dans le dossier de destination spécifié.

\sphinxstepscope


\chapter{Coeur Lucterios}
\label{\detokenize{CORE/index:coeur-lucterios}}\label{\detokenize{CORE/index::doc}}
\sphinxAtStartPar
Aide relative aux fonctionnalités générales de cet outil de gestion.

\sphinxstepscope


\section{Mot de passe}
\label{\detokenize{CORE/password:mot-de-passe}}\label{\detokenize{CORE/password::doc}}\begin{quote}

\sphinxAtStartPar
Menu \sphinxstyleemphasis{Général/Mot de passe}
\end{quote}

\sphinxAtStartPar
Vous pouvez changer le mot de passe d’accès de l’utilisateur courant.

\noindent{\hspace*{\fill}\sphinxincludegraphics[height=400\sphinxpxdimen]{{password}.png}\hspace*{\fill}}

\sphinxAtStartPar
Pour plus de sécurité, nous vous conseillons d’utiliser un mot de passe comprenant des lettres et des chiffres et ne constituant pas un mot compréhensible.

\sphinxstepscope


\section{Les groupes}
\label{\detokenize{CORE/groups:les-groupes}}\label{\detokenize{CORE/groups::doc}}\begin{quote}

\sphinxAtStartPar
Menu \sphinxstyleemphasis{Administration/Gestion des Droits/Les groupes}
\end{quote}

\sphinxAtStartPar
Créez, modifiez ou supprimez un groupe de droits.

\noindent{\hspace*{\fill}\sphinxincludegraphics[height=400\sphinxpxdimen]{{group}.png}\hspace*{\fill}}

\sphinxAtStartPar
Un groupe de droits permet de définir les autorisations (accès à certaines fonctionnalités du logiciel) qui sont consenties aux utilisateurs de l’application rattachés à celui\sphinxhyphen{}ci.

\noindent{\hspace*{\fill}\sphinxincludegraphics[height=400\sphinxpxdimen]{{group_modify}.png}\hspace*{\fill}}

\sphinxstepscope


\section{Les utilisateurs}
\label{\detokenize{CORE/users:les-utilisateurs}}\label{\detokenize{CORE/users::doc}}\begin{quote}

\sphinxAtStartPar
Menu \sphinxstyleemphasis{Administration/Gestion des Droits/Les utilisateurs}
\end{quote}

\sphinxAtStartPar
Créez, modifiez ou désactivez un utilisateur de l’application. Chacun bénéficie d’un droit de connexion à \sphinxstyleemphasis{diacamma} dont vous définissez l’étendue.

\noindent{\hspace*{\fill}\sphinxincludegraphics[height=400\sphinxpxdimen]{{users}.png}\hspace*{\fill}}

\sphinxAtStartPar
Depuis cette liste, vous pouvez créer ou modifier un utilisateur : son alias, son nom et son mot de passe.

\noindent{\hspace*{\fill}\sphinxincludegraphics[height=400\sphinxpxdimen]{{user_info}.png}\hspace*{\fill}}

\sphinxAtStartPar
Vous pouvez aussi l’inscrire dans des groupes, lui accorder des permissions suplémentaires afin de définir son niveau d’accès au logiciel. Vous pouvez aussi désactiver un utilisateur pour lui interdire l’accès à l’application.

\noindent{\hspace*{\fill}\sphinxincludegraphics[height=400\sphinxpxdimen]{{user_permissions}.png}\hspace*{\fill}}

\sphinxstepscope


\section{L’architecture du logiciel}
\label{\detokenize{CORE/architecture:l-architecture-du-logiciel}}\label{\detokenize{CORE/architecture::doc}}
\sphinxAtStartPar
Depuis le commencement de ce logiciel, les développeurs ont voulu que cette application puisse avoir une architecture ouverte permettant des évolutions les plus larges.



\renewcommand{\indexname}{Index}
\printindex
\end{document}